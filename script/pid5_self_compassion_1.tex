% Options for packages loaded elsewhere
% Options for packages loaded elsewhere
\PassOptionsToPackage{unicode}{hyperref}
\PassOptionsToPackage{hyphens}{url}
\PassOptionsToPackage{dvipsnames,svgnames,x11names}{xcolor}
%
\documentclass[
  11pt,
  a4paper,
  onecolumn]{article}
\usepackage{xcolor}
\usepackage[top=1in,left=1in,right=1in,bottom=1in]{geometry}
\usepackage{amsmath,amssymb}
\setcounter{secnumdepth}{-\maxdimen} % remove section numbering
\usepackage{iftex}
\ifPDFTeX
  \usepackage[T1]{fontenc}
  \usepackage[utf8]{inputenc}
  \usepackage{textcomp} % provide euro and other symbols
\else % if luatex or xetex
  \usepackage{unicode-math} % this also loads fontspec
  \defaultfontfeatures{Scale=MatchLowercase}
  \defaultfontfeatures[\rmfamily]{Ligatures=TeX,Scale=1}
\fi
\usepackage{lmodern}
\ifPDFTeX\else
  % xetex/luatex font selection
\fi
% Use upquote if available, for straight quotes in verbatim environments
\IfFileExists{upquote.sty}{\usepackage{upquote}}{}
\IfFileExists{microtype.sty}{% use microtype if available
  \usepackage[]{microtype}
  \UseMicrotypeSet[protrusion]{basicmath} % disable protrusion for tt fonts
}{}
\usepackage{setspace}
\makeatletter
\@ifundefined{KOMAClassName}{% if non-KOMA class
  \IfFileExists{parskip.sty}{%
    \usepackage{parskip}
  }{% else
    \setlength{\parindent}{0pt}
    \setlength{\parskip}{6pt plus 2pt minus 1pt}}
}{% if KOMA class
  \KOMAoptions{parskip=half}}
\makeatother
% Make \paragraph and \subparagraph free-standing
\makeatletter
\ifx\paragraph\undefined\else
  \let\oldparagraph\paragraph
  \renewcommand{\paragraph}{
    \@ifstar
      \xxxParagraphStar
      \xxxParagraphNoStar
  }
  \newcommand{\xxxParagraphStar}[1]{\oldparagraph*{#1}\mbox{}}
  \newcommand{\xxxParagraphNoStar}[1]{\oldparagraph{#1}\mbox{}}
\fi
\ifx\subparagraph\undefined\else
  \let\oldsubparagraph\subparagraph
  \renewcommand{\subparagraph}{
    \@ifstar
      \xxxSubParagraphStar
      \xxxSubParagraphNoStar
  }
  \newcommand{\xxxSubParagraphStar}[1]{\oldsubparagraph*{#1}\mbox{}}
  \newcommand{\xxxSubParagraphNoStar}[1]{\oldsubparagraph{#1}\mbox{}}
\fi
\makeatother

\usepackage{color}
\usepackage{fancyvrb}
\newcommand{\VerbBar}{|}
\newcommand{\VERB}{\Verb[commandchars=\\\{\}]}
\DefineVerbatimEnvironment{Highlighting}{Verbatim}{commandchars=\\\{\}}
% Add ',fontsize=\small' for more characters per line
\newenvironment{Shaded}{}{}
\newcommand{\AlertTok}[1]{\textcolor[rgb]{1.00,0.33,0.33}{\textbf{#1}}}
\newcommand{\AnnotationTok}[1]{\textcolor[rgb]{0.42,0.45,0.49}{#1}}
\newcommand{\AttributeTok}[1]{\textcolor[rgb]{0.84,0.23,0.29}{#1}}
\newcommand{\BaseNTok}[1]{\textcolor[rgb]{0.00,0.36,0.77}{#1}}
\newcommand{\BuiltInTok}[1]{\textcolor[rgb]{0.84,0.23,0.29}{#1}}
\newcommand{\CharTok}[1]{\textcolor[rgb]{0.01,0.18,0.38}{#1}}
\newcommand{\CommentTok}[1]{\textcolor[rgb]{0.42,0.45,0.49}{#1}}
\newcommand{\CommentVarTok}[1]{\textcolor[rgb]{0.42,0.45,0.49}{#1}}
\newcommand{\ConstantTok}[1]{\textcolor[rgb]{0.00,0.36,0.77}{#1}}
\newcommand{\ControlFlowTok}[1]{\textcolor[rgb]{0.84,0.23,0.29}{#1}}
\newcommand{\DataTypeTok}[1]{\textcolor[rgb]{0.84,0.23,0.29}{#1}}
\newcommand{\DecValTok}[1]{\textcolor[rgb]{0.00,0.36,0.77}{#1}}
\newcommand{\DocumentationTok}[1]{\textcolor[rgb]{0.42,0.45,0.49}{#1}}
\newcommand{\ErrorTok}[1]{\textcolor[rgb]{1.00,0.33,0.33}{\underline{#1}}}
\newcommand{\ExtensionTok}[1]{\textcolor[rgb]{0.84,0.23,0.29}{\textbf{#1}}}
\newcommand{\FloatTok}[1]{\textcolor[rgb]{0.00,0.36,0.77}{#1}}
\newcommand{\FunctionTok}[1]{\textcolor[rgb]{0.44,0.26,0.76}{#1}}
\newcommand{\ImportTok}[1]{\textcolor[rgb]{0.01,0.18,0.38}{#1}}
\newcommand{\InformationTok}[1]{\textcolor[rgb]{0.42,0.45,0.49}{#1}}
\newcommand{\KeywordTok}[1]{\textcolor[rgb]{0.84,0.23,0.29}{#1}}
\newcommand{\NormalTok}[1]{\textcolor[rgb]{0.14,0.16,0.18}{#1}}
\newcommand{\OperatorTok}[1]{\textcolor[rgb]{0.14,0.16,0.18}{#1}}
\newcommand{\OtherTok}[1]{\textcolor[rgb]{0.44,0.26,0.76}{#1}}
\newcommand{\PreprocessorTok}[1]{\textcolor[rgb]{0.84,0.23,0.29}{#1}}
\newcommand{\RegionMarkerTok}[1]{\textcolor[rgb]{0.42,0.45,0.49}{#1}}
\newcommand{\SpecialCharTok}[1]{\textcolor[rgb]{0.00,0.36,0.77}{#1}}
\newcommand{\SpecialStringTok}[1]{\textcolor[rgb]{0.01,0.18,0.38}{#1}}
\newcommand{\StringTok}[1]{\textcolor[rgb]{0.01,0.18,0.38}{#1}}
\newcommand{\VariableTok}[1]{\textcolor[rgb]{0.89,0.38,0.04}{#1}}
\newcommand{\VerbatimStringTok}[1]{\textcolor[rgb]{0.01,0.18,0.38}{#1}}
\newcommand{\WarningTok}[1]{\textcolor[rgb]{1.00,0.33,0.33}{#1}}

\usepackage{longtable,booktabs,array}
\usepackage{calc} % for calculating minipage widths
% Correct order of tables after \paragraph or \subparagraph
\usepackage{etoolbox}
\makeatletter
\patchcmd\longtable{\par}{\if@noskipsec\mbox{}\fi\par}{}{}
\makeatother
% Allow footnotes in longtable head/foot
\IfFileExists{footnotehyper.sty}{\usepackage{footnotehyper}}{\usepackage{footnote}}
\makesavenoteenv{longtable}
\usepackage{graphicx}
\makeatletter
\newsavebox\pandoc@box
\newcommand*\pandocbounded[1]{% scales image to fit in text height/width
  \sbox\pandoc@box{#1}%
  \Gscale@div\@tempa{\textheight}{\dimexpr\ht\pandoc@box+\dp\pandoc@box\relax}%
  \Gscale@div\@tempb{\linewidth}{\wd\pandoc@box}%
  \ifdim\@tempb\p@<\@tempa\p@\let\@tempa\@tempb\fi% select the smaller of both
  \ifdim\@tempa\p@<\p@\scalebox{\@tempa}{\usebox\pandoc@box}%
  \else\usebox{\pandoc@box}%
  \fi%
}
% Set default figure placement to htbp
\def\fps@figure{htbp}
\makeatother





\setlength{\emergencystretch}{3em} % prevent overfull lines

\providecommand{\tightlist}{%
  \setlength{\itemsep}{0pt}\setlength{\parskip}{0pt}}



 


\makeatletter
\@ifpackageloaded{caption}{}{\usepackage{caption}}
\AtBeginDocument{%
\ifdefined\contentsname
  \renewcommand*\contentsname{Table of contents}
\else
  \newcommand\contentsname{Table of contents}
\fi
\ifdefined\listfigurename
  \renewcommand*\listfigurename{List of Figures}
\else
  \newcommand\listfigurename{List of Figures}
\fi
\ifdefined\listtablename
  \renewcommand*\listtablename{List of Tables}
\else
  \newcommand\listtablename{List of Tables}
\fi
\ifdefined\figurename
  \renewcommand*\figurename{Figure}
\else
  \newcommand\figurename{Figure}
\fi
\ifdefined\tablename
  \renewcommand*\tablename{Table}
\else
  \newcommand\tablename{Table}
\fi
}
\@ifpackageloaded{float}{}{\usepackage{float}}
\floatstyle{ruled}
\@ifundefined{c@chapter}{\newfloat{codelisting}{h}{lop}}{\newfloat{codelisting}{h}{lop}[chapter]}
\floatname{codelisting}{Listing}
\newcommand*\listoflistings{\listof{codelisting}{List of Listings}}
\makeatother
\makeatletter
\makeatother
\makeatletter
\@ifpackageloaded{caption}{}{\usepackage{caption}}
\@ifpackageloaded{subcaption}{}{\usepackage{subcaption}}
\makeatother
\usepackage{bookmark}
\IfFileExists{xurl.sty}{\usepackage{xurl}}{} % add URL line breaks if available
\urlstyle{same}
\hypersetup{
  pdftitle={Static PID-5 and EMA Self-Compassion},
  pdfauthor={Corrado Caudek},
  colorlinks=true,
  linkcolor={blue},
  filecolor={Maroon},
  citecolor={Blue},
  urlcolor={Blue},
  pdfcreator={LaTeX via pandoc}}


\title{Static PID-5 and EMA Self-Compassion}
\author{Corrado Caudek}
\date{}
\begin{document}
\maketitle


\setstretch{1}
Le misure ``basali'' corrispondenti ai 5 domini del PID-5 sono state
calcolate \textbf{escludendo} i 15 item che vengono usati nelle
notifiche EMA.

\begin{Shaded}
\begin{Highlighting}[]
\CommentTok{\# Read and process \textquotesingle{}esi\_bf\textquotesingle{} data}
\NormalTok{esi\_bf }\OtherTok{\textless{}{-}}\NormalTok{ rio}\SpecialCharTok{::}\FunctionTok{import}\NormalTok{(}
\NormalTok{  here}\SpecialCharTok{::}\FunctionTok{here}\NormalTok{(}
    \StringTok{"data"}\NormalTok{,}
    \StringTok{"processed"}\NormalTok{,}
    \StringTok{"esi\_bf.csv"}
\NormalTok{  )}
\NormalTok{) }\SpecialCharTok{|\textgreater{}}
\NormalTok{  dplyr}\SpecialCharTok{::}\FunctionTok{distinct}\NormalTok{(user\_id, }\AttributeTok{.keep\_all =} \ConstantTok{TRUE}\NormalTok{) }\SpecialCharTok{|\textgreater{}} \CommentTok{\# Keep only distinct user\_id}
\NormalTok{  dplyr}\SpecialCharTok{::}\FunctionTok{select}\NormalTok{(user\_id, esi\_bf) }\CommentTok{\# Select relevant columns}

\CommentTok{\# Read and process \textquotesingle{}pid5\textquotesingle{} data}
\NormalTok{pid5 }\OtherTok{\textless{}{-}}\NormalTok{ rio}\SpecialCharTok{::}\FunctionTok{import}\NormalTok{(}
\NormalTok{  here}\SpecialCharTok{::}\FunctionTok{here}\NormalTok{(}
    \StringTok{"data"}\NormalTok{,}
    \StringTok{"processed"}\NormalTok{,}
    \StringTok{"pid5.csv"}
\NormalTok{  )}
\NormalTok{) }\SpecialCharTok{|\textgreater{}}
\NormalTok{  dplyr}\SpecialCharTok{::}\FunctionTok{distinct}\NormalTok{(user\_id, }\AttributeTok{.keep\_all =} \ConstantTok{TRUE}\NormalTok{) }\SpecialCharTok{|\textgreater{}}  \CommentTok{\# Keep only distinct user\_id}
\NormalTok{  dplyr}\SpecialCharTok{::}\FunctionTok{select}\NormalTok{(user\_id, }\FunctionTok{starts\_with}\NormalTok{(}\StringTok{"domain\_"}\NormalTok{)) }\CommentTok{\# Select domain variables}

\CommentTok{\# Merge \textquotesingle{}esi\_bf\textquotesingle{} and \textquotesingle{}pid5\textquotesingle{} data by user\_id}
\NormalTok{df }\OtherTok{\textless{}{-}} \FunctionTok{left\_join}\NormalTok{(esi\_bf, pid5, }\AttributeTok{by =} \StringTok{"user\_id"}\NormalTok{)}
\end{Highlighting}
\end{Shaded}

\begin{Shaded}
\begin{Highlighting}[]
\CommentTok{\# Define list of user IDs with careless responding}
\NormalTok{user\_id\_with\_careless\_responding }\OtherTok{\textless{}{-}} \FunctionTok{c}\NormalTok{(}
  \StringTok{"ma\_se\_2005\_11\_14\_490"}\NormalTok{,}
  \StringTok{"reve20041021036"}\NormalTok{,}
  \StringTok{"di\_ma\_2005\_10\_20\_756"}\NormalTok{,}
  \StringTok{"pa\_sc\_2005\_09\_10\_468"}\NormalTok{,}
  \StringTok{"il\_re\_2006\_01\_18\_645"}\NormalTok{,}
  \StringTok{"so\_ma\_2003\_10\_13\_804"}\NormalTok{,}
  \StringTok{"lo\_ca\_2005\_05\_07\_05\_437"}\NormalTok{,}
  \StringTok{"va\_ma\_2005\_05\_31\_567"}\NormalTok{,}
  \StringTok{"no\_un\_2005\_06\_29\_880"}\NormalTok{,}
  \StringTok{"an\_bo\_1988\_08\_24\_166"}\NormalTok{,}
  \StringTok{"st\_ma\_2004\_04\_21\_426"}\NormalTok{,}
  \StringTok{"an\_st\_2005\_10\_16\_052"}\NormalTok{,}
  \StringTok{"vi\_de\_2002\_12\_30\_067"}\NormalTok{,}
  \StringTok{"gi\_ru\_2005\_03\_08\_033"}\NormalTok{,}
  \StringTok{"al\_mi\_2005\_03\_05\_844"}\NormalTok{,}
  \StringTok{"la\_ma\_2006\_01\_31\_787"}\NormalTok{,}
  \StringTok{"gi\_lo\_2004\_06\_27\_237"}\NormalTok{,}
  \StringTok{"ch\_bi\_2001\_01\_28\_407"}\NormalTok{,}
  \StringTok{"al\_pe\_2001\_04\_20\_079"}\NormalTok{,}
  \StringTok{"le\_de\_2003\_09\_05\_067"}\NormalTok{,}
  \StringTok{"fe\_gr\_2002\_02\_19\_434"}\NormalTok{,}
  \StringTok{"ma\_ba\_2002\_09\_09\_052"}\NormalTok{,}
  \StringTok{"ca\_gi\_2003\_09\_16\_737"}\NormalTok{,}
  \StringTok{"an\_to\_2003\_08\_06\_114"}\NormalTok{,}
  \StringTok{"al\_se\_2003\_07\_28\_277"}\NormalTok{,}
  \StringTok{"ja\_tr\_2002\_10\_06\_487"}\NormalTok{,}
  \StringTok{"el\_ci\_2002\_02\_15\_057"}\NormalTok{,}
  \StringTok{"se\_ti\_2000\_03\_04\_975"}\NormalTok{,}
  \StringTok{"co\_ga\_2003\_10\_29\_614"}\NormalTok{,}
  \StringTok{"al\_ba\_2003\_18\_07\_905"}\NormalTok{,}
  \StringTok{"bi\_ro\_2003\_09\_07\_934"}\NormalTok{,}
  \StringTok{"an\_va\_2004\_04\_08\_527"}\NormalTok{,}
  \StringTok{"ev\_cr\_2003\_01\_27\_573"}
\NormalTok{)}

\CommentTok{\# Filter out users with careless responses}
\NormalTok{df1 }\OtherTok{\textless{}{-}}\NormalTok{ df[}\SpecialCharTok{!}\NormalTok{(df}\SpecialCharTok{$}\NormalTok{user\_id }\SpecialCharTok{\%in\%}\NormalTok{ user\_id\_with\_careless\_responding), ]}
\end{Highlighting}
\end{Shaded}

\begin{Shaded}
\begin{Highlighting}[]
\CommentTok{\# Read EMA data and rename \textquotesingle{}subj\_code\textquotesingle{} to \textquotesingle{}user\_id\textquotesingle{}}
\NormalTok{ema\_raw }\OtherTok{\textless{}{-}} \FunctionTok{readRDS}\NormalTok{(}
\NormalTok{  here}\SpecialCharTok{::}\FunctionTok{here}\NormalTok{(}
    \StringTok{"data"}\NormalTok{,}
    \StringTok{"raw"}\NormalTok{,}
    \StringTok{"ema"}\NormalTok{,}
    \StringTok{"ema\_data\_scoring.RDS"}
\NormalTok{  )}
\NormalTok{) }\SpecialCharTok{|\textgreater{}}
\NormalTok{  dplyr}\SpecialCharTok{::}\FunctionTok{rename}\NormalTok{(}
    \AttributeTok{user\_id =}\NormalTok{ subj\_code}
\NormalTok{  )}

\CommentTok{\# Merge EMA data with filtered main data}
\NormalTok{df2 }\OtherTok{\textless{}{-}} \FunctionTok{left\_join}\NormalTok{(df1, ema\_raw, }\AttributeTok{by =} \StringTok{"user\_id"}\NormalTok{)}

\CommentTok{\# Verify number of unique users}
\FunctionTok{length}\NormalTok{(}\FunctionTok{unique}\NormalTok{(df2}\SpecialCharTok{$}\NormalTok{user\_id))}
\end{Highlighting}
\end{Shaded}

\begin{verbatim}
[1] 429
\end{verbatim}

\subsection{Compliance}\label{compliance}

Escludiamo i soggetti che hanno risposto a meno di 10 notifiche.

\begin{Shaded}
\begin{Highlighting}[]
\CommentTok{\# Conta quante risposte EMA ha fornito ciascun soggetto}
\NormalTok{user\_counts }\OtherTok{\textless{}{-}}\NormalTok{ df2 }\SpecialCharTok{\%\textgreater{}\%}
  \FunctionTok{group\_by}\NormalTok{(user\_id) }\SpecialCharTok{\%\textgreater{}\%}
  \FunctionTok{summarise}\NormalTok{(}\AttributeTok{n\_responses =} \FunctionTok{n}\NormalTok{()) }\SpecialCharTok{\%\textgreater{}\%}
  \FunctionTok{ungroup}\NormalTok{()}

\CommentTok{\# Tieni solo i soggetti con almeno 10 risposte}
\NormalTok{valid\_users }\OtherTok{\textless{}{-}}\NormalTok{ user\_counts }\SpecialCharTok{\%\textgreater{}\%}
  \FunctionTok{filter}\NormalTok{(n\_responses }\SpecialCharTok{\textgreater{}=} \DecValTok{10}\NormalTok{) }\SpecialCharTok{\%\textgreater{}\%}
  \FunctionTok{pull}\NormalTok{(user\_id)}

\CommentTok{\# Filtra il dataframe originale}
\NormalTok{df2 }\OtherTok{\textless{}{-}}\NormalTok{ df2 }\SpecialCharTok{\%\textgreater{}\%}
\NormalTok{  dplyr}\SpecialCharTok{::}\FunctionTok{filter}\NormalTok{(user\_id }\SpecialCharTok{\%in\%}\NormalTok{ valid\_users)}
\end{Highlighting}
\end{Shaded}

\begin{Shaded}
\begin{Highlighting}[]
\FunctionTok{length}\NormalTok{(}\FunctionTok{unique}\NormalTok{(df2}\SpecialCharTok{$}\NormalTok{user\_id))}
\end{Highlighting}
\end{Shaded}

\begin{verbatim}
[1] 379
\end{verbatim}

\subsection{Generate negative instant
mood}\label{generate-negative-instant-mood}

\begin{Shaded}
\begin{Highlighting}[]
\CommentTok{\# Costruisce una misura media dell\textquotesingle{}affetto negativo momentaneo}

\CommentTok{\# Seleziona solo le colonne rilevanti (per velocità)}
\NormalTok{items }\OtherTok{\textless{}{-}} \FunctionTok{c}\NormalTok{(}\StringTok{"sad"}\NormalTok{, }\StringTok{"angry"}\NormalTok{, }\StringTok{"happy"}\NormalTok{, }\StringTok{"satisfied"}\NormalTok{)}

\CommentTok{\# Imputa i missing (1 solo imputazione, dato che i NA sono pochi)}
\NormalTok{imputed }\OtherTok{\textless{}{-}} \FunctionTok{mice}\NormalTok{(df2[, items], }\AttributeTok{m =} \DecValTok{1}\NormalTok{, }\AttributeTok{maxit =} \DecValTok{10}\NormalTok{, }\AttributeTok{seed =} \DecValTok{123}\NormalTok{)}
\end{Highlighting}
\end{Shaded}

\begin{verbatim}

 iter imp variable
  1   1  sad  angry  happy  satisfied
  2   1  sad  angry  happy  satisfied
  3   1  sad  angry  happy  satisfied
  4   1  sad  angry  happy  satisfied
  5   1  sad  angry  happy  satisfied
  6   1  sad  angry  happy  satisfied
  7   1  sad  angry  happy  satisfied
  8   1  sad  angry  happy  satisfied
  9   1  sad  angry  happy  satisfied
  10   1  sad  angry  happy  satisfied
\end{verbatim}

\begin{Shaded}
\begin{Highlighting}[]
\CommentTok{\# Estrai il dataset imputato e sostituisci le colonne originali}
\NormalTok{df2\_imputed }\OtherTok{\textless{}{-}} \FunctionTok{complete}\NormalTok{(imputed)}
\NormalTok{df2[, items] }\OtherTok{\textless{}{-}}\NormalTok{ df2\_imputed[, items]}

\NormalTok{df2 }\OtherTok{\textless{}{-}}\NormalTok{ df2 }\SpecialCharTok{\%\textgreater{}\%}
  \FunctionTok{mutate}\NormalTok{(}
    \AttributeTok{happy\_reversed =} \DecValTok{100} \SpecialCharTok{{-}}\NormalTok{ happy, }\CommentTok{\# Scala 0{-}100}
    \AttributeTok{satisfied\_reversed =} \DecValTok{100} \SpecialCharTok{{-}}\NormalTok{ satisfied,}
    \AttributeTok{neg\_aff\_ema =} \FunctionTok{rowMeans}\NormalTok{(}
      \FunctionTok{cbind}\NormalTok{(sad, angry, happy\_reversed, satisfied\_reversed),}
      \AttributeTok{na.rm =} \ConstantTok{TRUE}
\NormalTok{    )}
\NormalTok{  )}
\end{Highlighting}
\end{Shaded}

\subsection{Self-compassion negativa}\label{self-compassion-negativa}

Consideriamo solo le notifiche dove Self-Compassion è stata misurata.

\begin{Shaded}
\begin{Highlighting}[]
\NormalTok{df\_self\_comp\_ema }\OtherTok{\textless{}{-}}\NormalTok{ df2 }\SpecialCharTok{\%\textgreater{}\%}
\NormalTok{  dplyr}\SpecialCharTok{::}\FunctionTok{filter}\NormalTok{(}\SpecialCharTok{!}\FunctionTok{is.na}\NormalTok{(ucs\_neg) }\SpecialCharTok{\&} \SpecialCharTok{!}\FunctionTok{is.na}\NormalTok{(cs\_pos))}

\FunctionTok{length}\NormalTok{(}\FunctionTok{unique}\NormalTok{(df\_self\_comp\_ema}\SpecialCharTok{$}\NormalTok{user\_id))}
\end{Highlighting}
\end{Shaded}

\begin{verbatim}
[1] 379
\end{verbatim}

\begin{Shaded}
\begin{Highlighting}[]
\FunctionTok{dim}\NormalTok{(df\_self\_comp\_ema)}
\end{Highlighting}
\end{Shaded}

\begin{verbatim}
[1] 6229   92
\end{verbatim}

\begin{Shaded}
\begin{Highlighting}[]
\NormalTok{df\_self\_comp\_ema\_scaled }\OtherTok{\textless{}{-}}\NormalTok{ df\_self\_comp\_ema }\SpecialCharTok{\%\textgreater{}\%}
\NormalTok{  dplyr}\SpecialCharTok{::}\FunctionTok{select}\NormalTok{(}
\NormalTok{    ucs\_neg,}
\NormalTok{    domain\_negative\_affect,   }
\NormalTok{    domain\_detachment,}
\NormalTok{    domain\_antagonism,}
\NormalTok{    domain\_disinhibition,}
\NormalTok{    domain\_psychoticism,}
\NormalTok{    neg\_aff\_ema,}
\NormalTok{    pid5\_negative\_affectivity,}
\NormalTok{    pid5\_detachment,}
\NormalTok{    pid5\_antagonism,}
\NormalTok{    pid5\_disinhibition,}
\NormalTok{    pid5\_psychoticism,}
\NormalTok{    user\_id }\CommentTok{\# Mantiene user\_id così com\textquotesingle{}è}
\NormalTok{  ) }\SpecialCharTok{\%\textgreater{}\%}
\NormalTok{  dplyr}\SpecialCharTok{::}\FunctionTok{mutate}\NormalTok{(}
    \CommentTok{\# Applica la standardizzazione (scale) a tutte le colonne selezionate}
    \CommentTok{\# tranne user\_id. as.vector() è usato per assicurare che l\textquotesingle{}output sia un vettore.}
\NormalTok{    dplyr}\SpecialCharTok{::}\FunctionTok{across}\NormalTok{(}
      \FunctionTok{c}\NormalTok{(}
\NormalTok{        ucs\_neg,}
\NormalTok{        neg\_aff\_ema,}
\NormalTok{        domain\_negative\_affect,   }
\NormalTok{        domain\_detachment,}
\NormalTok{        domain\_antagonism,}
\NormalTok{        domain\_disinhibition,}
\NormalTok{        domain\_psychoticism,}
\NormalTok{        pid5\_negative\_affectivity,}
\NormalTok{        pid5\_detachment,}
\NormalTok{        pid5\_antagonism,}
\NormalTok{        pid5\_disinhibition,}
\NormalTok{        pid5\_psychoticism}
\NormalTok{      ),}
      \SpecialCharTok{\textasciitilde{}} \FunctionTok{as.vector}\NormalTok{(}\FunctionTok{scale}\NormalTok{(.))}
\NormalTok{    )}
\NormalTok{  )}
\end{Highlighting}
\end{Shaded}

\begin{Shaded}
\begin{Highlighting}[]
\NormalTok{model\_base }\OtherTok{\textless{}{-}} \FunctionTok{brm}\NormalTok{(}
\NormalTok{  ucs\_neg }\SpecialCharTok{\textasciitilde{}}\NormalTok{ neg\_aff\_ema }\SpecialCharTok{+} 
\NormalTok{    domain\_negative\_affect }\SpecialCharTok{+}\NormalTok{ domain\_detachment }\SpecialCharTok{+}
\NormalTok{    domain\_antagonism }\SpecialCharTok{+}\NormalTok{ domain\_disinhibition }\SpecialCharTok{+}\NormalTok{ domain\_psychoticism }\SpecialCharTok{+} 
\NormalTok{    (}\DecValTok{1} \SpecialCharTok{+}\NormalTok{ neg\_aff\_ema }\SpecialCharTok{|}\NormalTok{ user\_id),}
  \AttributeTok{data =}\NormalTok{ df\_self\_comp\_ema\_scaled,}
  \AttributeTok{family =} \FunctionTok{skew\_normal}\NormalTok{(),}
  \AttributeTok{prior =} \FunctionTok{c}\NormalTok{(}
    \FunctionTok{prior}\NormalTok{(}\FunctionTok{normal}\NormalTok{(}\DecValTok{0}\NormalTok{, }\DecValTok{1}\NormalTok{), }\AttributeTok{class =} \StringTok{"Intercept"}\NormalTok{),}
    \FunctionTok{prior}\NormalTok{(}\FunctionTok{normal}\NormalTok{(}\DecValTok{0}\NormalTok{, }\DecValTok{1}\NormalTok{), }\AttributeTok{class =} \StringTok{"b"}\NormalTok{),}
    \FunctionTok{prior}\NormalTok{(}\FunctionTok{exponential}\NormalTok{(}\DecValTok{1}\NormalTok{), }\AttributeTok{class =} \StringTok{"sd"}\NormalTok{),}
    \FunctionTok{prior}\NormalTok{(}\FunctionTok{exponential}\NormalTok{(}\DecValTok{1}\NormalTok{), }\AttributeTok{class =} \StringTok{"sigma"}\NormalTok{)}
\NormalTok{  ),}
  \AttributeTok{chains =} \DecValTok{4}\NormalTok{,}
  \AttributeTok{cores =} \DecValTok{4}\NormalTok{,}
  \AttributeTok{iter =} \DecValTok{2000}\NormalTok{,}
  \AttributeTok{seed =} \DecValTok{123}\NormalTok{,}
  \AttributeTok{backend =} \StringTok{"cmdstanr"}\NormalTok{,}
  \AttributeTok{save\_pars =} \FunctionTok{save\_pars}\NormalTok{(}\AttributeTok{all =} \ConstantTok{TRUE}\NormalTok{)}
\NormalTok{)}
\end{Highlighting}
\end{Shaded}

\begin{Shaded}
\begin{Highlighting}[]
\CommentTok{\# Posterior predictive check for the baseline model}
\FunctionTok{pp\_check}\NormalTok{(model\_base)}
\end{Highlighting}
\end{Shaded}

\begin{verbatim}
Using 10 posterior draws for ppc type 'dens_overlay' by default.
\end{verbatim}

\pandocbounded{\includegraphics[keepaspectratio]{pid5_self_compassion_1_files/figure-pdf/unnamed-chunk-12-1.pdf}}

\begin{Shaded}
\begin{Highlighting}[]
\FunctionTok{print}\NormalTok{(model\_base)}
\end{Highlighting}
\end{Shaded}

\begin{verbatim}
 Family: skew_normal 
  Links: mu = identity; sigma = identity; alpha = identity 
Formula: ucs_neg ~ neg_aff_ema + domain_negative_affect + domain_detachment + domain_antagonism + domain_disinhibition + domain_psychoticism + (1 + neg_aff_ema | user_id) 
   Data: df_self_comp_ema_scaled (Number of observations: 5757) 
  Draws: 4 chains, each with iter = 2000; warmup = 1000; thin = 1;
         total post-warmup draws = 4000

Multilevel Hyperparameters:
~user_id (Number of levels: 350) 
                           Estimate Est.Error l-95% CI u-95% CI Rhat Bulk_ESS
sd(Intercept)                  0.52      0.02     0.48     0.56 1.00      675
sd(neg_aff_ema)                0.21      0.02     0.18     0.24 1.00     1281
cor(Intercept,neg_aff_ema)     0.15      0.08    -0.01     0.30 1.00      970
                           Tail_ESS
sd(Intercept)                  1509
sd(neg_aff_ema)                2646
cor(Intercept,neg_aff_ema)     1932

Regression Coefficients:
                       Estimate Est.Error l-95% CI u-95% CI Rhat Bulk_ESS
Intercept                 -0.02      0.03    -0.08     0.04 1.01      367
neg_aff_ema                0.36      0.02     0.33     0.39 1.00     1525
domain_negative_affect     0.32      0.04     0.25     0.39 1.01      426
domain_detachment          0.05      0.03    -0.01     0.12 1.00      492
domain_antagonism          0.01      0.03    -0.06     0.07 1.01      383
domain_disinhibition       0.09      0.04     0.02     0.16 1.01      476
domain_psychoticism        0.01      0.04    -0.08     0.09 1.01      387
                       Tail_ESS
Intercept                   641
neg_aff_ema                2500
domain_negative_affect     1075
domain_detachment          1094
domain_antagonism           784
domain_disinhibition        921
domain_psychoticism         630

Further Distributional Parameters:
      Estimate Est.Error l-95% CI u-95% CI Rhat Bulk_ESS Tail_ESS
sigma     0.58      0.01     0.56     0.59 1.00     4834     3022
alpha     1.28      0.11     1.05     1.50 1.00     3136     3110

Draws were sampled using sample(hmc). For each parameter, Bulk_ESS
and Tail_ESS are effective sample size measures, and Rhat is the potential
scale reduction factor on split chains (at convergence, Rhat = 1).
\end{verbatim}

\begin{Shaded}
\begin{Highlighting}[]
\CommentTok{\# Fit augmented Bayesian model with interaction effects}
\NormalTok{model\_alt }\OtherTok{\textless{}{-}} \FunctionTok{brm}\NormalTok{(}
\NormalTok{  ucs\_neg }\SpecialCharTok{\textasciitilde{}}
\NormalTok{    (neg\_aff\_ema }\SpecialCharTok{+}\NormalTok{ domain\_negative\_affect }\SpecialCharTok{+}\NormalTok{ domain\_detachment }\SpecialCharTok{+} 
\NormalTok{       domain\_antagonism }\SpecialCharTok{+}\NormalTok{ domain\_disinhibition }\SpecialCharTok{+}\NormalTok{ domain\_psychoticism) }\SpecialCharTok{*}
\NormalTok{      (pid5\_negative\_affectivity }\SpecialCharTok{+}\NormalTok{ pid5\_detachment }\SpecialCharTok{+}\NormalTok{ pid5\_antagonism }\SpecialCharTok{+}
\NormalTok{         pid5\_disinhibition }\SpecialCharTok{+}\NormalTok{ pid5\_psychoticism) }\SpecialCharTok{+}
\NormalTok{    (}\DecValTok{1} \SpecialCharTok{+}\NormalTok{ neg\_aff\_ema }\SpecialCharTok{|}\NormalTok{ user\_id),}
  \AttributeTok{data =}\NormalTok{ df\_self\_comp\_ema\_scaled,}
  \AttributeTok{family =} \FunctionTok{skew\_normal}\NormalTok{(),}
  \AttributeTok{prior =} \FunctionTok{c}\NormalTok{(}
    \FunctionTok{prior}\NormalTok{(}\FunctionTok{normal}\NormalTok{(}\DecValTok{0}\NormalTok{, }\DecValTok{1}\NormalTok{), }\AttributeTok{class =} \StringTok{"Intercept"}\NormalTok{),}
    \FunctionTok{prior}\NormalTok{(}\FunctionTok{normal}\NormalTok{(}\DecValTok{0}\NormalTok{, }\DecValTok{1}\NormalTok{), }\AttributeTok{class =} \StringTok{"b"}\NormalTok{),}
    \FunctionTok{prior}\NormalTok{(}\FunctionTok{exponential}\NormalTok{(}\DecValTok{1}\NormalTok{), }\AttributeTok{class =} \StringTok{"sd"}\NormalTok{),}
    \FunctionTok{prior}\NormalTok{(}\FunctionTok{exponential}\NormalTok{(}\DecValTok{1}\NormalTok{), }\AttributeTok{class =} \StringTok{"sigma"}\NormalTok{)}
\NormalTok{  ),}
  \AttributeTok{chains =} \DecValTok{4}\NormalTok{,}
  \AttributeTok{cores =} \DecValTok{4}\NormalTok{,}
  \AttributeTok{iter =} \DecValTok{2000}\NormalTok{,}
  \AttributeTok{seed =} \DecValTok{123}\NormalTok{,}
  \AttributeTok{backend =} \StringTok{"cmdstanr"}\NormalTok{,}
  \AttributeTok{save\_pars =} \FunctionTok{save\_pars}\NormalTok{(}\AttributeTok{all =} \ConstantTok{TRUE}\NormalTok{)}
\NormalTok{)}
\end{Highlighting}
\end{Shaded}

\begin{Shaded}
\begin{Highlighting}[]
\FunctionTok{pp\_check}\NormalTok{(model\_alt)}
\end{Highlighting}
\end{Shaded}

\begin{verbatim}
Using 10 posterior draws for ppc type 'dens_overlay' by default.
\end{verbatim}

\pandocbounded{\includegraphics[keepaspectratio]{pid5_self_compassion_1_files/figure-pdf/unnamed-chunk-15-1.pdf}}

\begin{Shaded}
\begin{Highlighting}[]
\FunctionTok{print}\NormalTok{(model\_alt)}
\end{Highlighting}
\end{Shaded}

\begin{verbatim}
 Family: skew_normal 
  Links: mu = identity; sigma = identity; alpha = identity 
Formula: ucs_neg ~ (neg_aff_ema + domain_negative_affect + domain_detachment + domain_antagonism + domain_disinhibition + domain_psychoticism) * (pid5_negative_affectivity + pid5_detachment + pid5_antagonism + pid5_disinhibition + pid5_psychoticism) + (1 + neg_aff_ema | user_id) 
   Data: df_self_comp_ema_scaled (Number of observations: 5757) 
  Draws: 4 chains, each with iter = 2000; warmup = 1000; thin = 1;
         total post-warmup draws = 4000

Multilevel Hyperparameters:
~user_id (Number of levels: 350) 
                           Estimate Est.Error l-95% CI u-95% CI Rhat Bulk_ESS
sd(Intercept)                  0.39      0.02     0.35     0.42 1.00      978
sd(neg_aff_ema)                0.13      0.01     0.10     0.16 1.00     1318
cor(Intercept,neg_aff_ema)     0.23      0.10     0.04     0.41 1.00     1915
                           Tail_ESS
sd(Intercept)                  1602
sd(neg_aff_ema)                2170
cor(Intercept,neg_aff_ema)     2860

Regression Coefficients:
                                                 Estimate Est.Error l-95% CI
Intercept                                           -0.03      0.02    -0.08
neg_aff_ema                                          0.19      0.01     0.16
domain_negative_affect                               0.21      0.03     0.15
domain_detachment                                    0.03      0.03    -0.02
domain_antagonism                                    0.00      0.03    -0.05
domain_disinhibition                                 0.05      0.03    -0.01
domain_psychoticism                                 -0.02      0.03    -0.08
pid5_negative_affectivity                            0.28      0.01     0.26
pid5_detachment                                      0.13      0.01     0.10
pid5_antagonism                                     -0.09      0.01    -0.12
pid5_disinhibition                                   0.15      0.01     0.13
pid5_psychoticism                                    0.04      0.02     0.01
neg_aff_ema:pid5_negative_affectivity               -0.00      0.01    -0.02
neg_aff_ema:pid5_detachment                         -0.02      0.01    -0.04
neg_aff_ema:pid5_antagonism                         -0.01      0.01    -0.03
neg_aff_ema:pid5_disinhibition                       0.03      0.01     0.02
neg_aff_ema:pid5_psychoticism                       -0.01      0.01    -0.04
domain_negative_affect:pid5_negative_affectivity     0.06      0.01     0.03
domain_negative_affect:pid5_detachment               0.02      0.01    -0.01
domain_negative_affect:pid5_antagonism              -0.02      0.01    -0.05
domain_negative_affect:pid5_disinhibition           -0.03      0.01    -0.06
domain_negative_affect:pid5_psychoticism            -0.02      0.02    -0.05
domain_detachment:pid5_negative_affectivity          0.01      0.01    -0.01
domain_detachment:pid5_detachment                    0.00      0.01    -0.03
domain_detachment:pid5_antagonism                    0.01      0.01    -0.02
domain_detachment:pid5_disinhibition                -0.01      0.01    -0.03
domain_detachment:pid5_psychoticism                 -0.00      0.01    -0.03
domain_antagonism:pid5_negative_affectivity         -0.00      0.01    -0.03
domain_antagonism:pid5_detachment                   -0.02      0.01    -0.05
domain_antagonism:pid5_antagonism                    0.03      0.01     0.01
domain_antagonism:pid5_disinhibition                -0.02      0.01    -0.04
domain_antagonism:pid5_psychoticism                 -0.01      0.01    -0.03
domain_disinhibition:pid5_negative_affectivity      -0.01      0.01    -0.04
domain_disinhibition:pid5_detachment                 0.00      0.01    -0.03
domain_disinhibition:pid5_antagonism                -0.02      0.01    -0.05
domain_disinhibition:pid5_disinhibition              0.02      0.01    -0.01
domain_disinhibition:pid5_psychoticism              -0.01      0.01    -0.04
domain_psychoticism:pid5_negative_affectivity        0.01      0.02    -0.02
domain_psychoticism:pid5_detachment                 -0.01      0.02    -0.04
domain_psychoticism:pid5_antagonism                 -0.00      0.01    -0.03
domain_psychoticism:pid5_disinhibition               0.01      0.02    -0.02
domain_psychoticism:pid5_psychoticism                0.01      0.02    -0.02
                                                 u-95% CI Rhat Bulk_ESS
Intercept                                            0.02 1.01      722
neg_aff_ema                                          0.21 1.00     2825
domain_negative_affect                               0.27 1.00      640
domain_detachment                                    0.08 1.01      637
domain_antagonism                                    0.05 1.02      599
domain_disinhibition                                 0.10 1.01      684
domain_psychoticism                                  0.04 1.00      765
pid5_negative_affectivity                            0.31 1.00     3291
pid5_detachment                                      0.16 1.00     2747
pid5_antagonism                                     -0.07 1.00     3048
pid5_disinhibition                                   0.17 1.00     3360
pid5_psychoticism                                    0.06 1.00     3028
neg_aff_ema:pid5_negative_affectivity                0.02 1.00     2959
neg_aff_ema:pid5_detachment                         -0.00 1.00     4052
neg_aff_ema:pid5_antagonism                          0.01 1.00     3919
neg_aff_ema:pid5_disinhibition                       0.05 1.00     4263
neg_aff_ema:pid5_psychoticism                        0.01 1.00     3163
domain_negative_affect:pid5_negative_affectivity     0.08 1.00     2659
domain_negative_affect:pid5_detachment               0.05 1.00     2712
domain_negative_affect:pid5_antagonism               0.00 1.00     3231
domain_negative_affect:pid5_disinhibition           -0.01 1.00     3573
domain_negative_affect:pid5_psychoticism             0.02 1.00     2406
domain_detachment:pid5_negative_affectivity          0.04 1.00     2468
domain_detachment:pid5_detachment                    0.03 1.00     2825
domain_detachment:pid5_antagonism                    0.03 1.00     3485
domain_detachment:pid5_disinhibition                 0.02 1.00     3260
domain_detachment:pid5_psychoticism                  0.02 1.00     2815
domain_antagonism:pid5_negative_affectivity          0.02 1.00     2785
domain_antagonism:pid5_detachment                    0.00 1.00     2496
domain_antagonism:pid5_antagonism                    0.06 1.00     2756
domain_antagonism:pid5_disinhibition                 0.01 1.00     3556
domain_antagonism:pid5_psychoticism                  0.02 1.00     2712
domain_disinhibition:pid5_negative_affectivity       0.02 1.00     2792
domain_disinhibition:pid5_detachment                 0.03 1.00     2828
domain_disinhibition:pid5_antagonism                -0.00 1.00     3266
domain_disinhibition:pid5_disinhibition              0.04 1.00     3486
domain_disinhibition:pid5_psychoticism               0.02 1.00     3082
domain_psychoticism:pid5_negative_affectivity        0.04 1.00     2214
domain_psychoticism:pid5_detachment                  0.02 1.00     2664
domain_psychoticism:pid5_antagonism                  0.02 1.00     2428
domain_psychoticism:pid5_disinhibition               0.04 1.00     2845
domain_psychoticism:pid5_psychoticism                0.04 1.00     2246
                                                 Tail_ESS
Intercept                                            1312
neg_aff_ema                                          3241
domain_negative_affect                               1249
domain_detachment                                    1367
domain_antagonism                                    1116
domain_disinhibition                                 1268
domain_psychoticism                                  1510
pid5_negative_affectivity                            3625
pid5_detachment                                      3248
pid5_antagonism                                      2582
pid5_disinhibition                                   3169
pid5_psychoticism                                    3049
neg_aff_ema:pid5_negative_affectivity                2924
neg_aff_ema:pid5_detachment                          3159
neg_aff_ema:pid5_antagonism                          3311
neg_aff_ema:pid5_disinhibition                       3167
neg_aff_ema:pid5_psychoticism                        3000
domain_negative_affect:pid5_negative_affectivity     2427
domain_negative_affect:pid5_detachment               3237
domain_negative_affect:pid5_antagonism               3026
domain_negative_affect:pid5_disinhibition            3109
domain_negative_affect:pid5_psychoticism             2961
domain_detachment:pid5_negative_affectivity          2975
domain_detachment:pid5_detachment                    3483
domain_detachment:pid5_antagonism                    3088
domain_detachment:pid5_disinhibition                 2828
domain_detachment:pid5_psychoticism                  3004
domain_antagonism:pid5_negative_affectivity          2687
domain_antagonism:pid5_detachment                    2927
domain_antagonism:pid5_antagonism                    3124
domain_antagonism:pid5_disinhibition                 3162
domain_antagonism:pid5_psychoticism                  3055
domain_disinhibition:pid5_negative_affectivity       3277
domain_disinhibition:pid5_detachment                 3223
domain_disinhibition:pid5_antagonism                 3161
domain_disinhibition:pid5_disinhibition              3222
domain_disinhibition:pid5_psychoticism               3184
domain_psychoticism:pid5_negative_affectivity        2867
domain_psychoticism:pid5_detachment                  2920
domain_psychoticism:pid5_antagonism                  2981
domain_psychoticism:pid5_disinhibition               3168
domain_psychoticism:pid5_psychoticism                2878

Further Distributional Parameters:
      Estimate Est.Error l-95% CI u-95% CI Rhat Bulk_ESS Tail_ESS
sigma     0.53      0.01     0.52     0.54 1.00     4865     2944
alpha     1.20      0.11     0.98     1.42 1.00     3705     3458

Draws were sampled using sample(hmc). For each parameter, Bulk_ESS
and Tail_ESS are effective sample size measures, and Rhat is the potential
scale reduction factor on split chains (at convergence, Rhat = 1).
\end{verbatim}

\begin{Shaded}
\begin{Highlighting}[]
\NormalTok{loo0 }\OtherTok{\textless{}{-}} \FunctionTok{loo}\NormalTok{(model\_base, }\AttributeTok{save\_psis =} \ConstantTok{TRUE}\NormalTok{)}
\end{Highlighting}
\end{Shaded}

\begin{verbatim}
Warning: Found 13 observations with a pareto_k > 0.7 in model 'model_base'. We
recommend to set 'moment_match = TRUE' in order to perform moment matching for
problematic observations.
\end{verbatim}

\begin{Shaded}
\begin{Highlighting}[]
\NormalTok{loo1 }\OtherTok{\textless{}{-}} \FunctionTok{loo}\NormalTok{(model\_alt, }\AttributeTok{save\_psis =} \ConstantTok{TRUE}\NormalTok{)}
\end{Highlighting}
\end{Shaded}

\begin{verbatim}
Warning: Found 5 observations with a pareto_k > 0.7 in model 'model_alt'. We
recommend to set 'moment_match = TRUE' in order to perform moment matching for
problematic observations.
\end{verbatim}

\begin{Shaded}
\begin{Highlighting}[]
\FunctionTok{loo\_compare}\NormalTok{(loo0, loo1)}
\end{Highlighting}
\end{Shaded}

\begin{verbatim}
           elpd_diff se_diff
model_alt     0.0       0.0 
model_base -475.1      42.1 
\end{verbatim}

\subsubsection{Visualizzare ELPD\_diff}\label{visualizzare-elpd_diff}

Visualizzare dove il modello alternativo (model\_alt) migliora la
predizione rispetto al modello di base (model\_base), a livello di
soggetto.

\begin{Shaded}
\begin{Highlighting}[]
\CommentTok{\# Differenza pointwise tra i due modelli}
\NormalTok{elpd\_diff }\OtherTok{\textless{}{-}}\NormalTok{ loo0}\SpecialCharTok{$}\NormalTok{pointwise[, }\StringTok{"elpd\_loo"}\NormalTok{] }\SpecialCharTok{{-}}\NormalTok{ loo1}\SpecialCharTok{$}\NormalTok{pointwise[, }\StringTok{"elpd\_loo"}\NormalTok{]}
\end{Highlighting}
\end{Shaded}

\begin{Shaded}
\begin{Highlighting}[]
\CommentTok{\# Recupera i dati usati nel modello}
\NormalTok{model\_data }\OtherTok{\textless{}{-}}\NormalTok{ model\_base}\SpecialCharTok{$}\NormalTok{data}

\CommentTok{\# Aggiungi la colonna con la differenza di ELPD}
\NormalTok{model\_data}\SpecialCharTok{$}\NormalTok{elpd\_diff }\OtherTok{\textless{}{-}}\NormalTok{ elpd\_diff}
\end{Highlighting}
\end{Shaded}

\begin{Shaded}
\begin{Highlighting}[]
\NormalTok{subject\_diffs }\OtherTok{\textless{}{-}}\NormalTok{ model\_data }\SpecialCharTok{\%\textgreater{}\%}
  \FunctionTok{group\_by}\NormalTok{(user\_id) }\SpecialCharTok{\%\textgreater{}\%}
  \FunctionTok{summarise}\NormalTok{(}
    \AttributeTok{mean\_elpd\_diff =} \FunctionTok{mean}\NormalTok{(elpd\_diff, }\AttributeTok{na.rm =} \ConstantTok{TRUE}\NormalTok{),}
    \AttributeTok{se =} \FunctionTok{sd}\NormalTok{(elpd\_diff, }\AttributeTok{na.rm =} \ConstantTok{TRUE}\NormalTok{) }\SpecialCharTok{/} \FunctionTok{sqrt}\NormalTok{(}\FunctionTok{n}\NormalTok{())}
\NormalTok{  ) }\SpecialCharTok{\%\textgreater{}\%}
  \FunctionTok{arrange}\NormalTok{(mean\_elpd\_diff)}
\end{Highlighting}
\end{Shaded}

\begin{Shaded}
\begin{Highlighting}[]
\FunctionTok{ggplot}\NormalTok{(subject\_diffs, }\FunctionTok{aes}\NormalTok{(}\AttributeTok{x =} \FunctionTok{reorder}\NormalTok{(user\_id, mean\_elpd\_diff), }\AttributeTok{y =}\NormalTok{ mean\_elpd\_diff)) }\SpecialCharTok{+}
  \FunctionTok{geom\_point}\NormalTok{() }\SpecialCharTok{+}
  \FunctionTok{geom\_errorbar}\NormalTok{(}\FunctionTok{aes}\NormalTok{(}\AttributeTok{ymin =}\NormalTok{ mean\_elpd\_diff }\SpecialCharTok{{-}}\NormalTok{ se, }\AttributeTok{ymax =}\NormalTok{ mean\_elpd\_diff }\SpecialCharTok{+}\NormalTok{ se),}
                \AttributeTok{width =} \FloatTok{0.2}\NormalTok{, }\AttributeTok{alpha =} \FloatTok{0.3}\NormalTok{) }\SpecialCharTok{+}
  \FunctionTok{geom\_hline}\NormalTok{(}\AttributeTok{yintercept =} \DecValTok{0}\NormalTok{, }\AttributeTok{linetype =} \StringTok{"dashed"}\NormalTok{) }\SpecialCharTok{+}
  \FunctionTok{coord\_flip}\NormalTok{() }\SpecialCharTok{+}
  \FunctionTok{labs}\NormalTok{(}\AttributeTok{title =} \StringTok{"ELPD difference by subject"}\NormalTok{,}
       \AttributeTok{x =} \StringTok{"user\_id (ordered)"}\NormalTok{,}
       \AttributeTok{y =} \StringTok{"ELPD(model\_base) {-} ELPD(model\_alt)"}\NormalTok{) }\SpecialCharTok{+}
  \FunctionTok{theme\_minimal}\NormalTok{() }\SpecialCharTok{+}
  \FunctionTok{scale\_x\_discrete}\NormalTok{(}\AttributeTok{labels =} \ConstantTok{NULL}\NormalTok{)}
\end{Highlighting}
\end{Shaded}

\pandocbounded{\includegraphics[keepaspectratio]{pid5_self_compassion_1_files/figure-pdf/unnamed-chunk-21-1.pdf}}

Ogni punto rappresenta un soggetto. L'asse y mostra la differenza di
ELPD tra i modelli: ELPD\_base − ELPD\_alt. I valori sotto lo zero
indicano che il modello alternativo predice meglio per quel soggetto. Le
barre di errore indicano l'incertezza (errore standard) per ciascun
soggetto. Nel caso presente, dato il valore complessivo di elpd\_diff =
-466, ci aspettiamo che la maggior parte dei soggetti abbia valori
negativi.

\begin{Shaded}
\begin{Highlighting}[]
\NormalTok{subject\_diffs }\SpecialCharTok{\%\textgreater{}\%}
  \FunctionTok{summarise}\NormalTok{(}
    \AttributeTok{n =} \FunctionTok{n}\NormalTok{(),}
    \AttributeTok{n\_better\_alt =} \FunctionTok{sum}\NormalTok{(mean\_elpd\_diff }\SpecialCharTok{\textless{}} \DecValTok{0}\NormalTok{),}
    \AttributeTok{proportion =}\NormalTok{ n\_better\_alt }\SpecialCharTok{/}\NormalTok{ n,}
    \AttributeTok{percent =}\NormalTok{ proportion }\SpecialCharTok{*} \DecValTok{100}
\NormalTok{  )}
\end{Highlighting}
\end{Shaded}

\begin{verbatim}
# A tibble: 1 x 4
      n n_better_alt proportion percent
  <int>        <int>      <dbl>   <dbl>
1   350          263      0.751    75.1
\end{verbatim}

Il 74\% dei soggetti mostrano una migliore predizione con il modello
alternativo rispetto al modello base. La preferenza per model\_alt è
quindi generalizzata, non guidata da pochi individui.

\begin{Shaded}
\begin{Highlighting}[]
\FunctionTok{ggplot}\NormalTok{(subject\_diffs, }\FunctionTok{aes}\NormalTok{(}\AttributeTok{x =}\NormalTok{ mean\_elpd\_diff)) }\SpecialCharTok{+}
  \FunctionTok{geom\_histogram}\NormalTok{(}\AttributeTok{bins =} \DecValTok{30}\NormalTok{, }\AttributeTok{fill =} \StringTok{"steelblue"}\NormalTok{, }\AttributeTok{color =} \StringTok{"white"}\NormalTok{) }\SpecialCharTok{+}
  \FunctionTok{geom\_vline}\NormalTok{(}\AttributeTok{xintercept =} \DecValTok{0}\NormalTok{, }\AttributeTok{linetype =} \StringTok{"dashed"}\NormalTok{) }\SpecialCharTok{+}
  \FunctionTok{labs}\NormalTok{(}
    \AttributeTok{title =} \StringTok{"Distribuzione delle differenze di ELPD"}\NormalTok{,}
    \AttributeTok{x =} \StringTok{"ELPD(model\_base) − ELPD(model\_alt)"}\NormalTok{,}
    \AttributeTok{y =} \StringTok{"Numero di soggetti"}
\NormalTok{  ) }\SpecialCharTok{+}
  \FunctionTok{theme\_minimal}\NormalTok{()}
\end{Highlighting}
\end{Shaded}

\pandocbounded{\includegraphics[keepaspectratio]{pid5_self_compassion_1_files/figure-pdf/unnamed-chunk-23-1.pdf}}

\begin{Shaded}
\begin{Highlighting}[]
\FunctionTok{ggplot}\NormalTok{(subject\_diffs, }\FunctionTok{aes}\NormalTok{(}\AttributeTok{x =}\NormalTok{ mean\_elpd\_diff)) }\SpecialCharTok{+}
  \FunctionTok{geom\_density}\NormalTok{(}\AttributeTok{fill =} \StringTok{"skyblue"}\NormalTok{, }\AttributeTok{alpha =} \FloatTok{0.6}\NormalTok{) }\SpecialCharTok{+}
  \FunctionTok{geom\_vline}\NormalTok{(}\AttributeTok{xintercept =} \DecValTok{0}\NormalTok{, }\AttributeTok{linetype =} \StringTok{"dashed"}\NormalTok{) }\SpecialCharTok{+}
  \FunctionTok{geom\_vline}\NormalTok{(}\AttributeTok{xintercept =} \FunctionTok{quantile}\NormalTok{(subject\_diffs}\SpecialCharTok{$}\NormalTok{mean\_elpd\_diff, }\FloatTok{0.95}\NormalTok{), }\AttributeTok{color =} \StringTok{"red"}\NormalTok{) }\SpecialCharTok{+}
  \FunctionTok{labs}\NormalTok{(}\AttributeTok{title =} \StringTok{"Soggetti per cui il modello peggiora"}\NormalTok{,}
       \AttributeTok{subtitle =} \StringTok{"Valori oltre il 95° percentile evidenziati"}\NormalTok{,}
       \AttributeTok{x =} \StringTok{"mean\_elpd\_diff"}\NormalTok{, }\AttributeTok{y =} \StringTok{"Densità"}\NormalTok{) }\SpecialCharTok{+}
  \FunctionTok{theme\_minimal}\NormalTok{()}
\end{Highlighting}
\end{Shaded}

\pandocbounded{\includegraphics[keepaspectratio]{pid5_self_compassion_1_files/figure-pdf/unnamed-chunk-24-1.pdf}}

\begin{Shaded}
\begin{Highlighting}[]
\FunctionTok{bayes\_R2}\NormalTok{(model\_base)}
\end{Highlighting}
\end{Shaded}

\begin{verbatim}
    Estimate   Est.Error      Q2.5     Q97.5
R2 0.6737499 0.004462602 0.6649309 0.6821722
\end{verbatim}

\begin{Shaded}
\begin{Highlighting}[]
\FunctionTok{bayes\_R2}\NormalTok{(model\_alt)}
\end{Highlighting}
\end{Shaded}

\begin{verbatim}
    Estimate   Est.Error      Q2.5     Q97.5
R2 0.7224773 0.003742531 0.7149858 0.7297672
\end{verbatim}

\begin{Shaded}
\begin{Highlighting}[]
\CommentTok{\# K{-}fold cross{-}validation (e.g., 10 folds)}
\CommentTok{\# kfold\_base \textless{}{-} kfold(model\_base, K = 5, seed = 123)}
\CommentTok{\# kfold\_alt  \textless{}{-} kfold(model\_alt,  K = 5, seed = 123)}
\CommentTok{\# kfold\_compare(kfold\_base, kfold\_alt)}
\CommentTok{\# Se elpd\_diff è negativo per model\_base, vuol dire che model\_alt predice meglio }
\CommentTok{\# anche in validazione k{-}fold.}
\end{Highlighting}
\end{Shaded}

\begin{Shaded}
\begin{Highlighting}[]
\NormalTok{subject\_diffs }\OtherTok{\textless{}{-}}\NormalTok{ subject\_diffs }\SpecialCharTok{\%\textgreater{}\%}
  \FunctionTok{mutate}\NormalTok{(}\AttributeTok{benefit\_score =} \FunctionTok{scale}\NormalTok{(}\SpecialCharTok{{-}}\NormalTok{mean\_elpd\_diff)) }
\CommentTok{\# valori alti = miglioramento maggiore}
\NormalTok{subject\_diffs}
\end{Highlighting}
\end{Shaded}

\begin{verbatim}
# A tibble: 350 x 4
   user_id              mean_elpd_diff    se benefit_score[,1]
   <chr>                         <dbl> <dbl>             <dbl>
 1 so_li_2004_10_29_776         -1.21  0.361              6.35
 2 ch_va_2003_04_08_010         -1.03  0.525              5.31
 3 el_ca_2003_06_14_053         -0.849 0.297              4.32
 4 mi_lo_2005_03_17_960         -0.710 0.607              3.54
 5 gi_ma_2004_01_10_447         -0.690 0.502              3.42
 6 ca_fo_2002_08_30_071         -0.642 0.368              3.15
 7 an_gr_2003_02_23_266         -0.630 0.622              3.09
 8 al_ne_2005_11_07_247         -0.605 0.261              2.94
 9 an_ba_2003_04_19_988         -0.526 0.144              2.50
10 ir_mo_2005_02_23_157         -0.521 0.351              2.47
# i 340 more rows
\end{verbatim}

\subsubsection{Discussione dei risultati: impatto delle misure dinamiche
sui modelli
predittivi}\label{discussione-dei-risultati-impatto-delle-misure-dinamiche-sui-modelli-predittivi}

L'obiettivo principale di questa analisi era valutare se l'integrazione
delle \textbf{misure dinamiche dei tratti disadattivi di personalità}
(ovvero, le valutazioni settimanali del PID-5 tramite EMA) migliorasse
la capacità di prevedere l'intensità della \textbf{self-compassion
negativa} in risposta ad affetti negativi momentanei.

Per testare questa ipotesi, abbiamo confrontato due modelli:

\begin{itemize}
\tightlist
\item
  un \textbf{modello base}, in cui la self-compassion negativa (UCS) era
  spiegata da indicatori EMA dell'affetto negativo e dai tratti PID-5
  valutati una sola volta all'inizio dello studio;
\item
  un \textbf{modello alternativo}, in cui gli stessi predittori
  interagivano con le \textbf{misure EMA dei cinque domini PID-5},
  raccolte in parallelo ai dati di affetto negativo.
\end{itemize}

I risultati dell'analisi bayesiana con confronto via ELPD (Expected Log
Predictive Density) indicano un chiaro miglioramento nella predizione
per il modello che include le \textbf{interazioni con i tratti EMA}. In
particolare, la differenza complessiva di ELPD tra i modelli è di
\textbf{ΔELPD = -466}, a favore del modello alternativo. Questo effetto
non è guidato da pochi casi estremi: in oltre il \textbf{74\% dei
soggetti}, il modello con i tratti EMA ha fornito predizioni migliori, e
la distribuzione soggetto-specifica delle differenze di ELPD è
fortemente sbilanciata a favore del modello dinamico.

Anche la \textbf{varianza spiegata a posteriori (Bayes R²)} è maggiore
nel modello alternativo (R² = 0.52 vs.~0.41), suggerendo che la
variabilità intra-individuale nei tratti di personalità è un moderatore
cruciale della reattività affettiva momentanea.

Dal punto di vista teorico, questi risultati forniscono supporto
all'ipotesi che la relazione tra affetto negativo e self-compassion
negativa non sia una funzione stabile e fissa, ma \textbf{una funzione
modulata dai tratti di personalità così come si esprimono nel momento}.
L'uso delle misure EMA del PID-5 cattura queste \textbf{fluttuazioni
disposizionali contestuali}, che non sono accessibili tramite la sola
somministrazione statica del PID-5 a inizio studio.

In linea con un approccio \textbf{idionomico}, che mira a comprendere il
funzionamento individuale nel suo contesto situato, l'evidenza raccolta
suggerisce che \textbf{combinare misure di stato (affetto negativo
momentaneo) con misure di tratto dinamiche (PID-5 EMA)} permette una
modellazione più sensibile delle vulnerabilità psicopatologiche. Questi
risultati rafforzano l'idea che le valutazioni EMA non siano
semplicemente misure rumorose, ma rappresentino un valore aggiunto per
comprendere \textbf{quando} e \textbf{per chi} si attivano risposte
maladattive, come la self-compassion negativa.




\end{document}
