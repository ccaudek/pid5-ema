% Options for packages loaded elsewhere
% Options for packages loaded elsewhere
\PassOptionsToPackage{unicode}{hyperref}
\PassOptionsToPackage{hyphens}{url}
\PassOptionsToPackage{dvipsnames,svgnames,x11names}{xcolor}
%
\documentclass[
  11pt,
  a4paper,
  onecolumn]{article}
\usepackage{xcolor}
\usepackage[top=1in,left=1in,right=1in,bottom=1in]{geometry}
\usepackage{amsmath,amssymb}
\setcounter{secnumdepth}{-\maxdimen} % remove section numbering
\usepackage{iftex}
\ifPDFTeX
  \usepackage[T1]{fontenc}
  \usepackage[utf8]{inputenc}
  \usepackage{textcomp} % provide euro and other symbols
\else % if luatex or xetex
  \usepackage{unicode-math} % this also loads fontspec
  \defaultfontfeatures{Scale=MatchLowercase}
  \defaultfontfeatures[\rmfamily]{Ligatures=TeX,Scale=1}
\fi
\usepackage{lmodern}
\ifPDFTeX\else
  % xetex/luatex font selection
\fi
% Use upquote if available, for straight quotes in verbatim environments
\IfFileExists{upquote.sty}{\usepackage{upquote}}{}
\IfFileExists{microtype.sty}{% use microtype if available
  \usepackage[]{microtype}
  \UseMicrotypeSet[protrusion]{basicmath} % disable protrusion for tt fonts
}{}
\usepackage{setspace}
\makeatletter
\@ifundefined{KOMAClassName}{% if non-KOMA class
  \IfFileExists{parskip.sty}{%
    \usepackage{parskip}
  }{% else
    \setlength{\parindent}{0pt}
    \setlength{\parskip}{6pt plus 2pt minus 1pt}}
}{% if KOMA class
  \KOMAoptions{parskip=half}}
\makeatother
% Make \paragraph and \subparagraph free-standing
\makeatletter
\ifx\paragraph\undefined\else
  \let\oldparagraph\paragraph
  \renewcommand{\paragraph}{
    \@ifstar
      \xxxParagraphStar
      \xxxParagraphNoStar
  }
  \newcommand{\xxxParagraphStar}[1]{\oldparagraph*{#1}\mbox{}}
  \newcommand{\xxxParagraphNoStar}[1]{\oldparagraph{#1}\mbox{}}
\fi
\ifx\subparagraph\undefined\else
  \let\oldsubparagraph\subparagraph
  \renewcommand{\subparagraph}{
    \@ifstar
      \xxxSubParagraphStar
      \xxxSubParagraphNoStar
  }
  \newcommand{\xxxSubParagraphStar}[1]{\oldsubparagraph*{#1}\mbox{}}
  \newcommand{\xxxSubParagraphNoStar}[1]{\oldsubparagraph{#1}\mbox{}}
\fi
\makeatother

\usepackage{color}
\usepackage{fancyvrb}
\newcommand{\VerbBar}{|}
\newcommand{\VERB}{\Verb[commandchars=\\\{\}]}
\DefineVerbatimEnvironment{Highlighting}{Verbatim}{commandchars=\\\{\}}
% Add ',fontsize=\small' for more characters per line
\newenvironment{Shaded}{}{}
\newcommand{\AlertTok}[1]{\textcolor[rgb]{1.00,0.33,0.33}{\textbf{#1}}}
\newcommand{\AnnotationTok}[1]{\textcolor[rgb]{0.42,0.45,0.49}{#1}}
\newcommand{\AttributeTok}[1]{\textcolor[rgb]{0.84,0.23,0.29}{#1}}
\newcommand{\BaseNTok}[1]{\textcolor[rgb]{0.00,0.36,0.77}{#1}}
\newcommand{\BuiltInTok}[1]{\textcolor[rgb]{0.84,0.23,0.29}{#1}}
\newcommand{\CharTok}[1]{\textcolor[rgb]{0.01,0.18,0.38}{#1}}
\newcommand{\CommentTok}[1]{\textcolor[rgb]{0.42,0.45,0.49}{#1}}
\newcommand{\CommentVarTok}[1]{\textcolor[rgb]{0.42,0.45,0.49}{#1}}
\newcommand{\ConstantTok}[1]{\textcolor[rgb]{0.00,0.36,0.77}{#1}}
\newcommand{\ControlFlowTok}[1]{\textcolor[rgb]{0.84,0.23,0.29}{#1}}
\newcommand{\DataTypeTok}[1]{\textcolor[rgb]{0.84,0.23,0.29}{#1}}
\newcommand{\DecValTok}[1]{\textcolor[rgb]{0.00,0.36,0.77}{#1}}
\newcommand{\DocumentationTok}[1]{\textcolor[rgb]{0.42,0.45,0.49}{#1}}
\newcommand{\ErrorTok}[1]{\textcolor[rgb]{1.00,0.33,0.33}{\underline{#1}}}
\newcommand{\ExtensionTok}[1]{\textcolor[rgb]{0.84,0.23,0.29}{\textbf{#1}}}
\newcommand{\FloatTok}[1]{\textcolor[rgb]{0.00,0.36,0.77}{#1}}
\newcommand{\FunctionTok}[1]{\textcolor[rgb]{0.44,0.26,0.76}{#1}}
\newcommand{\ImportTok}[1]{\textcolor[rgb]{0.01,0.18,0.38}{#1}}
\newcommand{\InformationTok}[1]{\textcolor[rgb]{0.42,0.45,0.49}{#1}}
\newcommand{\KeywordTok}[1]{\textcolor[rgb]{0.84,0.23,0.29}{#1}}
\newcommand{\NormalTok}[1]{\textcolor[rgb]{0.14,0.16,0.18}{#1}}
\newcommand{\OperatorTok}[1]{\textcolor[rgb]{0.14,0.16,0.18}{#1}}
\newcommand{\OtherTok}[1]{\textcolor[rgb]{0.44,0.26,0.76}{#1}}
\newcommand{\PreprocessorTok}[1]{\textcolor[rgb]{0.84,0.23,0.29}{#1}}
\newcommand{\RegionMarkerTok}[1]{\textcolor[rgb]{0.42,0.45,0.49}{#1}}
\newcommand{\SpecialCharTok}[1]{\textcolor[rgb]{0.00,0.36,0.77}{#1}}
\newcommand{\SpecialStringTok}[1]{\textcolor[rgb]{0.01,0.18,0.38}{#1}}
\newcommand{\StringTok}[1]{\textcolor[rgb]{0.01,0.18,0.38}{#1}}
\newcommand{\VariableTok}[1]{\textcolor[rgb]{0.89,0.38,0.04}{#1}}
\newcommand{\VerbatimStringTok}[1]{\textcolor[rgb]{0.01,0.18,0.38}{#1}}
\newcommand{\WarningTok}[1]{\textcolor[rgb]{1.00,0.33,0.33}{#1}}

\usepackage{longtable,booktabs,array}
\usepackage{calc} % for calculating minipage widths
% Correct order of tables after \paragraph or \subparagraph
\usepackage{etoolbox}
\makeatletter
\patchcmd\longtable{\par}{\if@noskipsec\mbox{}\fi\par}{}{}
\makeatother
% Allow footnotes in longtable head/foot
\IfFileExists{footnotehyper.sty}{\usepackage{footnotehyper}}{\usepackage{footnote}}
\makesavenoteenv{longtable}
\usepackage{graphicx}
\makeatletter
\newsavebox\pandoc@box
\newcommand*\pandocbounded[1]{% scales image to fit in text height/width
  \sbox\pandoc@box{#1}%
  \Gscale@div\@tempa{\textheight}{\dimexpr\ht\pandoc@box+\dp\pandoc@box\relax}%
  \Gscale@div\@tempb{\linewidth}{\wd\pandoc@box}%
  \ifdim\@tempb\p@<\@tempa\p@\let\@tempa\@tempb\fi% select the smaller of both
  \ifdim\@tempa\p@<\p@\scalebox{\@tempa}{\usebox\pandoc@box}%
  \else\usebox{\pandoc@box}%
  \fi%
}
% Set default figure placement to htbp
\def\fps@figure{htbp}
\makeatother





\setlength{\emergencystretch}{3em} % prevent overfull lines

\providecommand{\tightlist}{%
  \setlength{\itemsep}{0pt}\setlength{\parskip}{0pt}}



 


\makeatletter
\@ifpackageloaded{caption}{}{\usepackage{caption}}
\AtBeginDocument{%
\ifdefined\contentsname
  \renewcommand*\contentsname{Table of contents}
\else
  \newcommand\contentsname{Table of contents}
\fi
\ifdefined\listfigurename
  \renewcommand*\listfigurename{List of Figures}
\else
  \newcommand\listfigurename{List of Figures}
\fi
\ifdefined\listtablename
  \renewcommand*\listtablename{List of Tables}
\else
  \newcommand\listtablename{List of Tables}
\fi
\ifdefined\figurename
  \renewcommand*\figurename{Figure}
\else
  \newcommand\figurename{Figure}
\fi
\ifdefined\tablename
  \renewcommand*\tablename{Table}
\else
  \newcommand\tablename{Table}
\fi
}
\@ifpackageloaded{float}{}{\usepackage{float}}
\floatstyle{ruled}
\@ifundefined{c@chapter}{\newfloat{codelisting}{h}{lop}}{\newfloat{codelisting}{h}{lop}[chapter]}
\floatname{codelisting}{Listing}
\newcommand*\listoflistings{\listof{codelisting}{List of Listings}}
\makeatother
\makeatletter
\makeatother
\makeatletter
\@ifpackageloaded{caption}{}{\usepackage{caption}}
\@ifpackageloaded{subcaption}{}{\usepackage{subcaption}}
\makeatother
\usepackage{bookmark}
\IfFileExists{xurl.sty}{\usepackage{xurl}}{} % add URL line breaks if available
\urlstyle{same}
\hypersetup{
  pdftitle={Static PID-5 and EMA Self-Compassion},
  pdfauthor={Corrado Caudek},
  colorlinks=true,
  linkcolor={blue},
  filecolor={Maroon},
  citecolor={Blue},
  urlcolor={Blue},
  pdfcreator={LaTeX via pandoc}}


\title{Static PID-5 and EMA Self-Compassion}
\author{Corrado Caudek}
\date{}
\begin{document}
\maketitle


\setstretch{1}
Le misure ``basali'' corrispondenti ai 5 domini del PID-5 sono state
calcolate \textbf{escludendo} i 15 item che vengono usati nelle
notifiche EMA.

\begin{Shaded}
\begin{Highlighting}[]
\CommentTok{\# Read and process \textquotesingle{}esi\_bf\textquotesingle{} data}
\NormalTok{esi\_bf }\OtherTok{\textless{}{-}}\NormalTok{ rio}\SpecialCharTok{::}\FunctionTok{import}\NormalTok{(}
\NormalTok{  here}\SpecialCharTok{::}\FunctionTok{here}\NormalTok{(}
    \StringTok{"data"}\NormalTok{,}
    \StringTok{"processed"}\NormalTok{,}
    \StringTok{"esi\_bf.csv"}
\NormalTok{  )}
\NormalTok{) }\SpecialCharTok{|\textgreater{}}
\NormalTok{  dplyr}\SpecialCharTok{::}\FunctionTok{distinct}\NormalTok{(user\_id, }\AttributeTok{.keep\_all =} \ConstantTok{TRUE}\NormalTok{) }\SpecialCharTok{|\textgreater{}} \CommentTok{\# Keep only distinct user\_id}
\NormalTok{  dplyr}\SpecialCharTok{::}\FunctionTok{select}\NormalTok{(user\_id, esi\_bf) }\CommentTok{\# Select relevant columns}

\CommentTok{\# Read and process \textquotesingle{}pid5\textquotesingle{} data}
\NormalTok{pid5 }\OtherTok{\textless{}{-}}\NormalTok{ rio}\SpecialCharTok{::}\FunctionTok{import}\NormalTok{(}
\NormalTok{  here}\SpecialCharTok{::}\FunctionTok{here}\NormalTok{(}
    \StringTok{"data"}\NormalTok{,}
    \StringTok{"processed"}\NormalTok{,}
    \StringTok{"pid5.csv"}
\NormalTok{  )}
\NormalTok{) }\SpecialCharTok{|\textgreater{}}
\NormalTok{  dplyr}\SpecialCharTok{::}\FunctionTok{distinct}\NormalTok{(user\_id, }\AttributeTok{.keep\_all =} \ConstantTok{TRUE}\NormalTok{) }\SpecialCharTok{|\textgreater{}}  \CommentTok{\# Keep only distinct user\_id}
\NormalTok{  dplyr}\SpecialCharTok{::}\FunctionTok{select}\NormalTok{(user\_id, }\FunctionTok{starts\_with}\NormalTok{(}\StringTok{"domain\_"}\NormalTok{)) }\CommentTok{\# Select domain variables}

\CommentTok{\# Merge \textquotesingle{}esi\_bf\textquotesingle{} and \textquotesingle{}pid5\textquotesingle{} data by user\_id}
\NormalTok{df }\OtherTok{\textless{}{-}} \FunctionTok{left\_join}\NormalTok{(esi\_bf, pid5, }\AttributeTok{by =} \StringTok{"user\_id"}\NormalTok{)}
\end{Highlighting}
\end{Shaded}

\begin{Shaded}
\begin{Highlighting}[]
\CommentTok{\# Define list of user IDs with careless responding}
\NormalTok{user\_id\_with\_careless\_responding }\OtherTok{\textless{}{-}} \FunctionTok{c}\NormalTok{(}
  \StringTok{"ma\_se\_2005\_11\_14\_490"}\NormalTok{,}
  \StringTok{"reve20041021036"}\NormalTok{,}
  \StringTok{"di\_ma\_2005\_10\_20\_756"}\NormalTok{,}
  \StringTok{"pa\_sc\_2005\_09\_10\_468"}\NormalTok{,}
  \StringTok{"il\_re\_2006\_01\_18\_645"}\NormalTok{,}
  \StringTok{"so\_ma\_2003\_10\_13\_804"}\NormalTok{,}
  \StringTok{"lo\_ca\_2005\_05\_07\_05\_437"}\NormalTok{,}
  \StringTok{"va\_ma\_2005\_05\_31\_567"}\NormalTok{,}
  \StringTok{"no\_un\_2005\_06\_29\_880"}\NormalTok{,}
  \StringTok{"an\_bo\_1988\_08\_24\_166"}\NormalTok{,}
  \StringTok{"st\_ma\_2004\_04\_21\_426"}\NormalTok{,}
  \StringTok{"an\_st\_2005\_10\_16\_052"}\NormalTok{,}
  \StringTok{"vi\_de\_2002\_12\_30\_067"}\NormalTok{,}
  \StringTok{"gi\_ru\_2005\_03\_08\_033"}\NormalTok{,}
  \StringTok{"al\_mi\_2005\_03\_05\_844"}\NormalTok{,}
  \StringTok{"la\_ma\_2006\_01\_31\_787"}\NormalTok{,}
  \StringTok{"gi\_lo\_2004\_06\_27\_237"}\NormalTok{,}
  \StringTok{"ch\_bi\_2001\_01\_28\_407"}\NormalTok{,}
  \StringTok{"al\_pe\_2001\_04\_20\_079"}\NormalTok{,}
  \StringTok{"le\_de\_2003\_09\_05\_067"}\NormalTok{,}
  \StringTok{"fe\_gr\_2002\_02\_19\_434"}\NormalTok{,}
  \StringTok{"ma\_ba\_2002\_09\_09\_052"}\NormalTok{,}
  \StringTok{"ca\_gi\_2003\_09\_16\_737"}\NormalTok{,}
  \StringTok{"an\_to\_2003\_08\_06\_114"}\NormalTok{,}
  \StringTok{"al\_se\_2003\_07\_28\_277"}\NormalTok{,}
  \StringTok{"ja\_tr\_2002\_10\_06\_487"}\NormalTok{,}
  \StringTok{"el\_ci\_2002\_02\_15\_057"}\NormalTok{,}
  \StringTok{"se\_ti\_2000\_03\_04\_975"}\NormalTok{,}
  \StringTok{"co\_ga\_2003\_10\_29\_614"}\NormalTok{,}
  \StringTok{"al\_ba\_2003\_18\_07\_905"}\NormalTok{,}
  \StringTok{"bi\_ro\_2003\_09\_07\_934"}\NormalTok{,}
  \StringTok{"an\_va\_2004\_04\_08\_527"}\NormalTok{,}
  \StringTok{"ev\_cr\_2003\_01\_27\_573"}
\NormalTok{)}

\CommentTok{\# Filter out users with careless responses}
\NormalTok{df1 }\OtherTok{\textless{}{-}}\NormalTok{ df[}\SpecialCharTok{!}\NormalTok{(df}\SpecialCharTok{$}\NormalTok{user\_id }\SpecialCharTok{\%in\%}\NormalTok{ user\_id\_with\_careless\_responding), ]}
\end{Highlighting}
\end{Shaded}

\begin{Shaded}
\begin{Highlighting}[]
\CommentTok{\# Read EMA data and rename \textquotesingle{}subj\_code\textquotesingle{} to \textquotesingle{}user\_id\textquotesingle{}}
\NormalTok{ema\_raw }\OtherTok{\textless{}{-}} \FunctionTok{readRDS}\NormalTok{(}
\NormalTok{  here}\SpecialCharTok{::}\FunctionTok{here}\NormalTok{(}
    \StringTok{"data"}\NormalTok{,}
    \StringTok{"raw"}\NormalTok{,}
    \StringTok{"ema"}\NormalTok{,}
    \StringTok{"ema\_data\_scoring.RDS"}
\NormalTok{  )}
\NormalTok{) }\SpecialCharTok{|\textgreater{}}
\NormalTok{  dplyr}\SpecialCharTok{::}\FunctionTok{rename}\NormalTok{(}
    \AttributeTok{user\_id =}\NormalTok{ subj\_code}
\NormalTok{  )}

\CommentTok{\# Merge EMA data with filtered main data}
\NormalTok{df2 }\OtherTok{\textless{}{-}} \FunctionTok{left\_join}\NormalTok{(df1, ema\_raw, }\AttributeTok{by =} \StringTok{"user\_id"}\NormalTok{)}

\CommentTok{\# Verify number of unique users}
\FunctionTok{length}\NormalTok{(}\FunctionTok{unique}\NormalTok{(df2}\SpecialCharTok{$}\NormalTok{user\_id))}
\end{Highlighting}
\end{Shaded}

\begin{verbatim}
[1] 429
\end{verbatim}

\subsection{Compliance}\label{compliance}

Escludiamo i soggetti che hanno risposto a meno di 10 notifiche.

\begin{Shaded}
\begin{Highlighting}[]
\CommentTok{\# Conta quante risposte EMA ha fornito ciascun soggetto}
\NormalTok{user\_counts }\OtherTok{\textless{}{-}}\NormalTok{ df2 }\SpecialCharTok{\%\textgreater{}\%}
  \FunctionTok{group\_by}\NormalTok{(user\_id) }\SpecialCharTok{\%\textgreater{}\%}
  \FunctionTok{summarise}\NormalTok{(}\AttributeTok{n\_responses =} \FunctionTok{n}\NormalTok{()) }\SpecialCharTok{\%\textgreater{}\%}
  \FunctionTok{ungroup}\NormalTok{()}

\CommentTok{\# Tieni solo i soggetti con almeno 10 risposte}
\NormalTok{valid\_users }\OtherTok{\textless{}{-}}\NormalTok{ user\_counts }\SpecialCharTok{\%\textgreater{}\%}
  \FunctionTok{filter}\NormalTok{(n\_responses }\SpecialCharTok{\textgreater{}=} \DecValTok{10}\NormalTok{) }\SpecialCharTok{\%\textgreater{}\%}
  \FunctionTok{pull}\NormalTok{(user\_id)}

\CommentTok{\# Filtra il dataframe originale}
\NormalTok{df2 }\OtherTok{\textless{}{-}}\NormalTok{ df2 }\SpecialCharTok{\%\textgreater{}\%}
\NormalTok{  dplyr}\SpecialCharTok{::}\FunctionTok{filter}\NormalTok{(user\_id }\SpecialCharTok{\%in\%}\NormalTok{ valid\_users)}
\end{Highlighting}
\end{Shaded}

\begin{Shaded}
\begin{Highlighting}[]
\FunctionTok{length}\NormalTok{(}\FunctionTok{unique}\NormalTok{(df2}\SpecialCharTok{$}\NormalTok{user\_id))}
\end{Highlighting}
\end{Shaded}

\begin{verbatim}
[1] 379
\end{verbatim}

\subsection{Generate negative instant
mood}\label{generate-negative-instant-mood}

\begin{Shaded}
\begin{Highlighting}[]
\CommentTok{\# Costruisce una misura media dell\textquotesingle{}affetto negativo momentaneo}

\CommentTok{\# Seleziona solo le colonne rilevanti (per velocità)}
\NormalTok{items }\OtherTok{\textless{}{-}} \FunctionTok{c}\NormalTok{(}\StringTok{"sad"}\NormalTok{, }\StringTok{"angry"}\NormalTok{, }\StringTok{"happy"}\NormalTok{, }\StringTok{"satisfied"}\NormalTok{)}

\CommentTok{\# Imputa i missing (1 solo imputazione, dato che i NA sono pochi)}
\NormalTok{imputed }\OtherTok{\textless{}{-}} \FunctionTok{mice}\NormalTok{(df2[, items], }\AttributeTok{m =} \DecValTok{1}\NormalTok{, }\AttributeTok{maxit =} \DecValTok{10}\NormalTok{, }\AttributeTok{seed =} \DecValTok{123}\NormalTok{)}
\end{Highlighting}
\end{Shaded}

\begin{verbatim}

 iter imp variable
  1   1  sad  angry  happy  satisfied
  2   1  sad  angry  happy  satisfied
  3   1  sad  angry  happy  satisfied
  4   1  sad  angry  happy  satisfied
  5   1  sad  angry  happy  satisfied
  6   1  sad  angry  happy  satisfied
  7   1  sad  angry  happy  satisfied
  8   1  sad  angry  happy  satisfied
  9   1  sad  angry  happy  satisfied
  10   1  sad  angry  happy  satisfied
\end{verbatim}

\begin{Shaded}
\begin{Highlighting}[]
\CommentTok{\# Estrai il dataset imputato e sostituisci le colonne originali}
\NormalTok{df2\_imputed }\OtherTok{\textless{}{-}} \FunctionTok{complete}\NormalTok{(imputed)}
\NormalTok{df2[, items] }\OtherTok{\textless{}{-}}\NormalTok{ df2\_imputed[, items]}

\NormalTok{df2 }\OtherTok{\textless{}{-}}\NormalTok{ df2 }\SpecialCharTok{\%\textgreater{}\%}
  \FunctionTok{mutate}\NormalTok{(}
    \AttributeTok{happy\_reversed =} \DecValTok{100} \SpecialCharTok{{-}}\NormalTok{ happy, }\CommentTok{\# Scala 0{-}100}
    \AttributeTok{satisfied\_reversed =} \DecValTok{100} \SpecialCharTok{{-}}\NormalTok{ satisfied,}
    \AttributeTok{neg\_aff\_ema =} \FunctionTok{rowMeans}\NormalTok{(}
      \FunctionTok{cbind}\NormalTok{(sad, angry, happy\_reversed, satisfied\_reversed),}
      \AttributeTok{na.rm =} \ConstantTok{TRUE}
\NormalTok{    )}
\NormalTok{  )}
\end{Highlighting}
\end{Shaded}

\subsection{Self-compassion negativa}\label{self-compassion-negativa}

Consideriamo solo le notifiche dove Self-Compassion è stata misurata.

\begin{Shaded}
\begin{Highlighting}[]
\NormalTok{df\_self\_comp\_ema }\OtherTok{\textless{}{-}}\NormalTok{ df2 }\SpecialCharTok{\%\textgreater{}\%}
\NormalTok{  dplyr}\SpecialCharTok{::}\FunctionTok{filter}\NormalTok{(}\SpecialCharTok{!}\FunctionTok{is.na}\NormalTok{(ucs\_neg) }\SpecialCharTok{\&} \SpecialCharTok{!}\FunctionTok{is.na}\NormalTok{(cs\_pos))}

\FunctionTok{length}\NormalTok{(}\FunctionTok{unique}\NormalTok{(df\_self\_comp\_ema}\SpecialCharTok{$}\NormalTok{user\_id))}
\end{Highlighting}
\end{Shaded}

\begin{verbatim}
[1] 379
\end{verbatim}

\begin{Shaded}
\begin{Highlighting}[]
\FunctionTok{dim}\NormalTok{(df\_self\_comp\_ema)}
\end{Highlighting}
\end{Shaded}

\begin{verbatim}
[1] 6229   92
\end{verbatim}

\begin{Shaded}
\begin{Highlighting}[]
\NormalTok{df\_self\_comp\_ema\_scaled }\OtherTok{\textless{}{-}}\NormalTok{ df\_self\_comp\_ema }\SpecialCharTok{\%\textgreater{}\%}
\NormalTok{  dplyr}\SpecialCharTok{::}\FunctionTok{select}\NormalTok{(}
\NormalTok{    ucs\_neg,}
\NormalTok{    domain\_negative\_affect,   }
\NormalTok{    domain\_detachment,}
\NormalTok{    domain\_antagonism,}
\NormalTok{    domain\_disinhibition,}
\NormalTok{    domain\_psychoticism,}
\NormalTok{    neg\_aff\_ema,}
\NormalTok{    pid5\_negative\_affectivity,}
\NormalTok{    pid5\_detachment,}
\NormalTok{    pid5\_antagonism,}
\NormalTok{    pid5\_disinhibition,}
\NormalTok{    pid5\_psychoticism,}
\NormalTok{    user\_id }\CommentTok{\# Mantiene user\_id così com\textquotesingle{}è}
\NormalTok{  ) }\SpecialCharTok{\%\textgreater{}\%}
\NormalTok{  dplyr}\SpecialCharTok{::}\FunctionTok{mutate}\NormalTok{(}
    \CommentTok{\# Applica la standardizzazione (scale) a tutte le colonne selezionate}
    \CommentTok{\# tranne user\_id. as.vector() è usato per assicurare che l\textquotesingle{}output sia un vettore.}
\NormalTok{    dplyr}\SpecialCharTok{::}\FunctionTok{across}\NormalTok{(}
      \FunctionTok{c}\NormalTok{(}
\NormalTok{        ucs\_neg,}
\NormalTok{        neg\_aff\_ema,}
\NormalTok{        domain\_negative\_affect,   }
\NormalTok{        domain\_detachment,}
\NormalTok{        domain\_antagonism,}
\NormalTok{        domain\_disinhibition,}
\NormalTok{        domain\_psychoticism,}
\NormalTok{        pid5\_negative\_affectivity,}
\NormalTok{        pid5\_detachment,}
\NormalTok{        pid5\_antagonism,}
\NormalTok{        pid5\_disinhibition,}
\NormalTok{        pid5\_psychoticism}
\NormalTok{      ),}
      \SpecialCharTok{\textasciitilde{}} \FunctionTok{as.vector}\NormalTok{(}\FunctionTok{scale}\NormalTok{(.))}
\NormalTok{    )}
\NormalTok{  )}
\end{Highlighting}
\end{Shaded}

\begin{Shaded}
\begin{Highlighting}[]
\NormalTok{model\_base }\OtherTok{\textless{}{-}} \FunctionTok{brm}\NormalTok{(}
\NormalTok{  ucs\_neg }\SpecialCharTok{\textasciitilde{}} \DecValTok{1} \SpecialCharTok{+}  
\NormalTok{    domain\_negative\_affect }\SpecialCharTok{+}\NormalTok{ domain\_detachment }\SpecialCharTok{+}
\NormalTok{    domain\_antagonism }\SpecialCharTok{+}\NormalTok{ domain\_disinhibition }\SpecialCharTok{+}\NormalTok{ domain\_psychoticism }\SpecialCharTok{+} 
\NormalTok{    (}\DecValTok{1} \SpecialCharTok{+}\NormalTok{ neg\_aff\_ema }\SpecialCharTok{|}\NormalTok{ user\_id),}
  \AttributeTok{data =}\NormalTok{ df\_self\_comp\_ema\_scaled,}
  \AttributeTok{family =} \FunctionTok{skew\_normal}\NormalTok{(),}
  \AttributeTok{prior =} \FunctionTok{c}\NormalTok{(}
    \FunctionTok{prior}\NormalTok{(}\FunctionTok{normal}\NormalTok{(}\DecValTok{0}\NormalTok{, }\DecValTok{1}\NormalTok{), }\AttributeTok{class =} \StringTok{"Intercept"}\NormalTok{),}
    \FunctionTok{prior}\NormalTok{(}\FunctionTok{normal}\NormalTok{(}\DecValTok{0}\NormalTok{, }\DecValTok{1}\NormalTok{), }\AttributeTok{class =} \StringTok{"b"}\NormalTok{),}
    \FunctionTok{prior}\NormalTok{(}\FunctionTok{exponential}\NormalTok{(}\DecValTok{1}\NormalTok{), }\AttributeTok{class =} \StringTok{"sd"}\NormalTok{),}
    \FunctionTok{prior}\NormalTok{(}\FunctionTok{exponential}\NormalTok{(}\DecValTok{1}\NormalTok{), }\AttributeTok{class =} \StringTok{"sigma"}\NormalTok{)}
\NormalTok{  ),}
  \AttributeTok{chains =} \DecValTok{4}\NormalTok{,}
  \AttributeTok{cores =} \DecValTok{4}\NormalTok{,}
  \AttributeTok{iter =} \DecValTok{2000}\NormalTok{,}
  \AttributeTok{seed =} \DecValTok{123}\NormalTok{,}
  \AttributeTok{backend =} \StringTok{"cmdstanr"}\NormalTok{,}
  \AttributeTok{save\_pars =} \FunctionTok{save\_pars}\NormalTok{(}\AttributeTok{all =} \ConstantTok{TRUE}\NormalTok{)}
\NormalTok{)}
\end{Highlighting}
\end{Shaded}

\begin{Shaded}
\begin{Highlighting}[]
\CommentTok{\# Posterior predictive check for the baseline model}
\FunctionTok{pp\_check}\NormalTok{(model\_base)}
\end{Highlighting}
\end{Shaded}

\begin{verbatim}
Using 10 posterior draws for ppc type 'dens_overlay' by default.
\end{verbatim}

\pandocbounded{\includegraphics[keepaspectratio]{pid5_self_compassion_1_files/figure-pdf/unnamed-chunk-12-1.pdf}}

\begin{Shaded}
\begin{Highlighting}[]
\FunctionTok{print}\NormalTok{(model\_base)}
\end{Highlighting}
\end{Shaded}

\begin{verbatim}
 Family: skew_normal 
  Links: mu = identity; sigma = identity; alpha = identity 
Formula: ucs_neg ~ 1 + domain_negative_affect + domain_detachment + domain_antagonism + domain_disinhibition + domain_psychoticism + (1 + neg_aff_ema | user_id) 
   Data: df_self_comp_ema_scaled (Number of observations: 5757) 
  Draws: 4 chains, each with iter = 2000; warmup = 1000; thin = 1;
         total post-warmup draws = 4000

Multilevel Hyperparameters:
~user_id (Number of levels: 350) 
                           Estimate Est.Error l-95% CI u-95% CI Rhat Bulk_ESS
sd(Intercept)                  0.55      0.02     0.50     0.60 1.00      699
sd(neg_aff_ema)                0.42      0.02     0.38     0.46 1.00     1135
cor(Intercept,neg_aff_ema)     0.24      0.09     0.06     0.42 1.03      176
                           Tail_ESS
sd(Intercept)                  1109
sd(neg_aff_ema)                1739
cor(Intercept,neg_aff_ema)      386

Regression Coefficients:
                       Estimate Est.Error l-95% CI u-95% CI Rhat Bulk_ESS
Intercept                 -0.13      0.05    -0.22    -0.04 1.04      157
domain_negative_affect     0.32      0.04     0.25     0.40 1.01      318
domain_detachment          0.06      0.03    -0.01     0.13 1.00      649
domain_antagonism         -0.00      0.03    -0.07     0.07 1.01      417
domain_disinhibition       0.09      0.04     0.02     0.17 1.01      503
domain_psychoticism        0.02      0.04    -0.06     0.10 1.01      595
                       Tail_ESS
Intercept                   285
domain_negative_affect      924
domain_detachment          1150
domain_antagonism           535
domain_disinhibition        791
domain_psychoticism        1100

Further Distributional Parameters:
      Estimate Est.Error l-95% CI u-95% CI Rhat Bulk_ESS Tail_ESS
sigma     0.57      0.01     0.56     0.59 1.00     4255     2986
alpha     1.39      0.11     1.17     1.62 1.00     3270     3077

Draws were sampled using sample(hmc). For each parameter, Bulk_ESS
and Tail_ESS are effective sample size measures, and Rhat is the potential
scale reduction factor on split chains (at convergence, Rhat = 1).
\end{verbatim}

\begin{Shaded}
\begin{Highlighting}[]
\CommentTok{\# Fit augmented Bayesian model with interaction effects}
\NormalTok{model\_alt }\OtherTok{\textless{}{-}} \FunctionTok{brm}\NormalTok{(}
\NormalTok{  ucs\_neg }\SpecialCharTok{\textasciitilde{}}
\NormalTok{    domain\_negative\_affect }\SpecialCharTok{*}\NormalTok{ pid5\_negative\_affectivity }\SpecialCharTok{+} 
\NormalTok{    domain\_detachment }\SpecialCharTok{*}\NormalTok{ pid5\_detachment }\SpecialCharTok{+}
\NormalTok{    domain\_antagonism }\SpecialCharTok{*}\NormalTok{ pid5\_antagonism }\SpecialCharTok{+} 
\NormalTok{    domain\_disinhibition }\SpecialCharTok{*}\NormalTok{ pid5\_disinhibition }\SpecialCharTok{+} 
\NormalTok{    domain\_psychoticism }\SpecialCharTok{*}\NormalTok{ pid5\_psychoticism }\SpecialCharTok{+}
\NormalTok{    (}\DecValTok{1} \SpecialCharTok{+}\NormalTok{ pid5\_negative\_affectivity }\SpecialCharTok{+}\NormalTok{ pid5\_detachment }\SpecialCharTok{+}\NormalTok{ pid5\_antagonism }\SpecialCharTok{+}
\NormalTok{       pid5\_disinhibition }\SpecialCharTok{+}\NormalTok{ pid5\_psychoticism }\SpecialCharTok{|}\NormalTok{ user\_id),}
  \AttributeTok{data =}\NormalTok{ df\_self\_comp\_ema\_scaled,}
  \AttributeTok{family =} \FunctionTok{skew\_normal}\NormalTok{(),}
  \AttributeTok{prior =} \FunctionTok{c}\NormalTok{(}
    \FunctionTok{prior}\NormalTok{(}\FunctionTok{normal}\NormalTok{(}\DecValTok{0}\NormalTok{, }\DecValTok{1}\NormalTok{), }\AttributeTok{class =} \StringTok{"Intercept"}\NormalTok{),}
    \FunctionTok{prior}\NormalTok{(}\FunctionTok{normal}\NormalTok{(}\DecValTok{0}\NormalTok{, }\DecValTok{1}\NormalTok{), }\AttributeTok{class =} \StringTok{"b"}\NormalTok{),}
    \FunctionTok{prior}\NormalTok{(}\FunctionTok{exponential}\NormalTok{(}\DecValTok{1}\NormalTok{), }\AttributeTok{class =} \StringTok{"sd"}\NormalTok{),}
    \FunctionTok{prior}\NormalTok{(}\FunctionTok{exponential}\NormalTok{(}\DecValTok{1}\NormalTok{), }\AttributeTok{class =} \StringTok{"sigma"}\NormalTok{)}
\NormalTok{  ),}
  \AttributeTok{chains =} \DecValTok{4}\NormalTok{,}
  \AttributeTok{cores =} \DecValTok{4}\NormalTok{,}
  \AttributeTok{iter =} \DecValTok{2000}\NormalTok{,}
  \CommentTok{\# seed = 123,}
  \AttributeTok{backend =} \StringTok{"cmdstanr"}\NormalTok{,}
  \AttributeTok{save\_pars =} \FunctionTok{save\_pars}\NormalTok{(}\AttributeTok{all =} \ConstantTok{TRUE}\NormalTok{)}
\NormalTok{)}
\end{Highlighting}
\end{Shaded}

\begin{Shaded}
\begin{Highlighting}[]
\FunctionTok{pp\_check}\NormalTok{(model\_alt)}
\end{Highlighting}
\end{Shaded}

\begin{verbatim}
Using 10 posterior draws for ppc type 'dens_overlay' by default.
\end{verbatim}

\pandocbounded{\includegraphics[keepaspectratio]{pid5_self_compassion_1_files/figure-pdf/unnamed-chunk-15-1.pdf}}

\begin{Shaded}
\begin{Highlighting}[]
\FunctionTok{print}\NormalTok{(model\_alt)}
\end{Highlighting}
\end{Shaded}

\begin{verbatim}
Warning: There were 2 divergent transitions after warmup. Increasing
adapt_delta above 0.8 may help. See
http://mc-stan.org/misc/warnings.html#divergent-transitions-after-warmup
\end{verbatim}

\begin{verbatim}
 Family: skew_normal 
  Links: mu = identity; sigma = identity; alpha = identity 
Formula: ucs_neg ~ domain_negative_affect * pid5_negative_affectivity + domain_detachment * pid5_detachment + domain_antagonism * pid5_antagonism + domain_disinhibition * pid5_disinhibition + domain_psychoticism * pid5_psychoticism + (1 + pid5_negative_affectivity + pid5_detachment + pid5_antagonism + pid5_disinhibition + pid5_psychoticism | user_id) 
   Data: df_self_comp_ema_scaled (Number of observations: 5757) 
  Draws: 4 chains, each with iter = 2000; warmup = 1000; thin = 1;
         total post-warmup draws = 4000

Multilevel Hyperparameters:
~user_id (Number of levels: 350) 
                                                  Estimate Est.Error l-95% CI
sd(Intercept)                                         0.38      0.02     0.35
sd(pid5_negative_affectivity)                         0.15      0.02     0.12
sd(pid5_detachment)                                   0.13      0.02     0.08
sd(pid5_antagonism)                                   0.11      0.02     0.06
sd(pid5_disinhibition)                                0.14      0.02     0.11
sd(pid5_psychoticism)                                 0.07      0.03     0.01
cor(Intercept,pid5_negative_affectivity)              0.18      0.11    -0.04
cor(Intercept,pid5_detachment)                       -0.09      0.13    -0.34
cor(pid5_negative_affectivity,pid5_detachment)       -0.08      0.19    -0.43
cor(Intercept,pid5_antagonism)                       -0.09      0.14    -0.35
cor(pid5_negative_affectivity,pid5_antagonism)        0.45      0.19     0.04
cor(pid5_detachment,pid5_antagonism)                 -0.14      0.25    -0.58
cor(Intercept,pid5_disinhibition)                    -0.06      0.11    -0.26
cor(pid5_negative_affectivity,pid5_disinhibition)    -0.18      0.16    -0.47
cor(pid5_detachment,pid5_disinhibition)              -0.28      0.20    -0.63
cor(pid5_antagonism,pid5_disinhibition)              -0.30      0.20    -0.68
cor(Intercept,pid5_psychoticism)                     -0.24      0.23    -0.64
cor(pid5_negative_affectivity,pid5_psychoticism)     -0.32      0.27    -0.76
cor(pid5_detachment,pid5_psychoticism)               -0.14      0.31    -0.70
cor(pid5_antagonism,pid5_psychoticism)                0.19      0.29    -0.43
cor(pid5_disinhibition,pid5_psychoticism)             0.04      0.28    -0.50
                                                  u-95% CI Rhat Bulk_ESS
sd(Intercept)                                         0.42 1.00     1393
sd(pid5_negative_affectivity)                         0.19 1.00     1315
sd(pid5_detachment)                                   0.17 1.01      686
sd(pid5_antagonism)                                   0.15 1.00      642
sd(pid5_disinhibition)                                0.17 1.00     1115
sd(pid5_psychoticism)                                 0.13 1.01      368
cor(Intercept,pid5_negative_affectivity)              0.39 1.00     2118
cor(Intercept,pid5_detachment)                        0.16 1.00     2510
cor(pid5_negative_affectivity,pid5_detachment)        0.31 1.00     1021
cor(Intercept,pid5_antagonism)                        0.19 1.00     2454
cor(pid5_negative_affectivity,pid5_antagonism)        0.79 1.00     1009
cor(pid5_detachment,pid5_antagonism)                  0.38 1.01      637
cor(Intercept,pid5_disinhibition)                     0.16 1.00     2726
cor(pid5_negative_affectivity,pid5_disinhibition)     0.16 1.00      938
cor(pid5_detachment,pid5_disinhibition)               0.14 1.01      452
cor(pid5_antagonism,pid5_disinhibition)               0.11 1.01      453
cor(Intercept,pid5_psychoticism)                      0.29 1.00     2807
cor(pid5_negative_affectivity,pid5_psychoticism)      0.31 1.00     1545
cor(pid5_detachment,pid5_psychoticism)                0.51 1.00     1061
cor(pid5_antagonism,pid5_psychoticism)                0.73 1.00     1313
cor(pid5_disinhibition,pid5_psychoticism)             0.58 1.00     1839
                                                  Tail_ESS
sd(Intercept)                                         2602
sd(pid5_negative_affectivity)                         2326
sd(pid5_detachment)                                    763
sd(pid5_antagonism)                                    838
sd(pid5_disinhibition)                                2449
sd(pid5_psychoticism)                                  990
cor(Intercept,pid5_negative_affectivity)              2690
cor(Intercept,pid5_detachment)                        2803
cor(pid5_negative_affectivity,pid5_detachment)        1378
cor(Intercept,pid5_antagonism)                        2685
cor(pid5_negative_affectivity,pid5_antagonism)        1768
cor(pid5_detachment,pid5_antagonism)                  1441
cor(Intercept,pid5_disinhibition)                     2861
cor(pid5_negative_affectivity,pid5_disinhibition)     1608
cor(pid5_detachment,pid5_disinhibition)                742
cor(pid5_antagonism,pid5_disinhibition)               1132
cor(Intercept,pid5_psychoticism)                      1836
cor(pid5_negative_affectivity,pid5_psychoticism)      2095
cor(pid5_detachment,pid5_psychoticism)                1877
cor(pid5_antagonism,pid5_psychoticism)                2077
cor(pid5_disinhibition,pid5_psychoticism)             2895

Regression Coefficients:
                                                 Estimate Est.Error l-95% CI
Intercept                                           -0.02      0.03    -0.06
domain_negative_affect                               0.23      0.03     0.17
pid5_negative_affectivity                            0.34      0.02     0.31
domain_detachment                                    0.03      0.03    -0.02
pid5_detachment                                      0.21      0.02     0.18
domain_antagonism                                    0.02      0.03    -0.03
pid5_antagonism                                     -0.11      0.02    -0.14
domain_disinhibition                                 0.03      0.03    -0.03
pid5_disinhibition                                   0.19      0.01     0.16
domain_psychoticism                                 -0.04      0.03    -0.11
pid5_psychoticism                                    0.02      0.02    -0.01
domain_negative_affect:pid5_negative_affectivity     0.06      0.01     0.04
domain_detachment:pid5_detachment                   -0.01      0.01    -0.04
domain_antagonism:pid5_antagonism                    0.01      0.01    -0.01
domain_disinhibition:pid5_disinhibition             -0.02      0.01    -0.04
domain_psychoticism:pid5_psychoticism               -0.01      0.01    -0.04
                                                 u-95% CI Rhat Bulk_ESS
Intercept                                            0.03 1.00     1605
domain_negative_affect                               0.28 1.00     1363
pid5_negative_affectivity                            0.37 1.00     3982
domain_detachment                                    0.08 1.00     1917
pid5_detachment                                      0.24 1.00     3938
domain_antagonism                                    0.08 1.00     1649
pid5_antagonism                                     -0.08 1.00     4286
domain_disinhibition                                 0.09 1.00     1418
pid5_disinhibition                                   0.22 1.00     4528
domain_psychoticism                                  0.02 1.00     1291
pid5_psychoticism                                    0.05 1.00     3686
domain_negative_affect:pid5_negative_affectivity     0.09 1.00     4054
domain_detachment:pid5_detachment                    0.01 1.00     3884
domain_antagonism:pid5_antagonism                    0.04 1.00     3657
domain_disinhibition:pid5_disinhibition              0.01 1.00     3206
domain_psychoticism:pid5_psychoticism                0.01 1.00     3959
                                                 Tail_ESS
Intercept                                            2312
domain_negative_affect                               1717
pid5_negative_affectivity                            3299
domain_detachment                                    2205
pid5_detachment                                      2932
domain_antagonism                                    2178
pid5_antagonism                                      3088
domain_disinhibition                                 2600
pid5_disinhibition                                   3066
domain_psychoticism                                  1796
pid5_psychoticism                                    2996
domain_negative_affect:pid5_negative_affectivity     3218
domain_detachment:pid5_detachment                    2666
domain_antagonism:pid5_antagonism                    2675
domain_disinhibition:pid5_disinhibition              2869
domain_psychoticism:pid5_psychoticism                3051

Further Distributional Parameters:
      Estimate Est.Error l-95% CI u-95% CI Rhat Bulk_ESS Tail_ESS
sigma     0.53      0.01     0.52     0.54 1.00     2675     2890
alpha     1.45      0.12     1.20     1.69 1.00     3844     2913

Draws were sampled using sample(hmc). For each parameter, Bulk_ESS
and Tail_ESS are effective sample size measures, and Rhat is the potential
scale reduction factor on split chains (at convergence, Rhat = 1).
\end{verbatim}

\begin{Shaded}
\begin{Highlighting}[]
\NormalTok{loo0 }\OtherTok{\textless{}{-}} \FunctionTok{loo}\NormalTok{(model\_base, }\AttributeTok{save\_psis =} \ConstantTok{TRUE}\NormalTok{)}
\end{Highlighting}
\end{Shaded}

\begin{verbatim}
Warning: Found 24 observations with a pareto_k > 0.7 in model 'model_base'. We
recommend to set 'moment_match = TRUE' in order to perform moment matching for
problematic observations.
\end{verbatim}

\begin{Shaded}
\begin{Highlighting}[]
\NormalTok{loo1 }\OtherTok{\textless{}{-}} \FunctionTok{loo}\NormalTok{(model\_alt, }\AttributeTok{save\_psis =} \ConstantTok{TRUE}\NormalTok{)}
\end{Highlighting}
\end{Shaded}

\begin{verbatim}
Warning: Found 20 observations with a pareto_k > 0.7 in model 'model_alt'. We
recommend to set 'moment_match = TRUE' in order to perform moment matching for
problematic observations.
\end{verbatim}

\begin{Shaded}
\begin{Highlighting}[]
\FunctionTok{loo\_compare}\NormalTok{(loo0, loo1)}
\end{Highlighting}
\end{Shaded}

\begin{verbatim}
           elpd_diff se_diff
model_alt     0.0       0.0 
model_base -416.8      55.7 
\end{verbatim}

\subsubsection{Visualizzare ELPD\_diff}\label{visualizzare-elpd_diff}

Visualizzare dove il modello alternativo (model\_alt) migliora la
predizione rispetto al modello di base (model\_base), a livello di
soggetto.

\begin{Shaded}
\begin{Highlighting}[]
\CommentTok{\# Differenza pointwise tra i due modelli}
\NormalTok{elpd\_diff }\OtherTok{\textless{}{-}}\NormalTok{ loo0}\SpecialCharTok{$}\NormalTok{pointwise[, }\StringTok{"elpd\_loo"}\NormalTok{] }\SpecialCharTok{{-}}\NormalTok{ loo1}\SpecialCharTok{$}\NormalTok{pointwise[, }\StringTok{"elpd\_loo"}\NormalTok{]}
\end{Highlighting}
\end{Shaded}

\begin{Shaded}
\begin{Highlighting}[]
\CommentTok{\# Recupera i dati usati nel modello}
\NormalTok{model\_data }\OtherTok{\textless{}{-}}\NormalTok{ model\_base}\SpecialCharTok{$}\NormalTok{data}

\CommentTok{\# Aggiungi la colonna con la differenza di ELPD}
\NormalTok{model\_data}\SpecialCharTok{$}\NormalTok{elpd\_diff }\OtherTok{\textless{}{-}}\NormalTok{ elpd\_diff}
\end{Highlighting}
\end{Shaded}

\begin{Shaded}
\begin{Highlighting}[]
\NormalTok{subject\_diffs }\OtherTok{\textless{}{-}}\NormalTok{ model\_data }\SpecialCharTok{\%\textgreater{}\%}
  \FunctionTok{group\_by}\NormalTok{(user\_id) }\SpecialCharTok{\%\textgreater{}\%}
  \FunctionTok{summarise}\NormalTok{(}
    \AttributeTok{mean\_elpd\_diff =} \FunctionTok{mean}\NormalTok{(elpd\_diff, }\AttributeTok{na.rm =} \ConstantTok{TRUE}\NormalTok{),}
    \AttributeTok{se =} \FunctionTok{sd}\NormalTok{(elpd\_diff, }\AttributeTok{na.rm =} \ConstantTok{TRUE}\NormalTok{) }\SpecialCharTok{/} \FunctionTok{sqrt}\NormalTok{(}\FunctionTok{n}\NormalTok{())}
\NormalTok{  ) }\SpecialCharTok{\%\textgreater{}\%}
  \FunctionTok{arrange}\NormalTok{(mean\_elpd\_diff)}
\end{Highlighting}
\end{Shaded}

\begin{Shaded}
\begin{Highlighting}[]
\FunctionTok{ggplot}\NormalTok{(subject\_diffs, }\FunctionTok{aes}\NormalTok{(}\AttributeTok{x =} \FunctionTok{reorder}\NormalTok{(user\_id, mean\_elpd\_diff), }\AttributeTok{y =}\NormalTok{ mean\_elpd\_diff)) }\SpecialCharTok{+}
  \FunctionTok{geom\_point}\NormalTok{() }\SpecialCharTok{+}
  \FunctionTok{geom\_errorbar}\NormalTok{(}\FunctionTok{aes}\NormalTok{(}\AttributeTok{ymin =}\NormalTok{ mean\_elpd\_diff }\SpecialCharTok{{-}}\NormalTok{ se, }\AttributeTok{ymax =}\NormalTok{ mean\_elpd\_diff }\SpecialCharTok{+}\NormalTok{ se),}
                \AttributeTok{width =} \FloatTok{0.2}\NormalTok{, }\AttributeTok{alpha =} \FloatTok{0.3}\NormalTok{) }\SpecialCharTok{+}
  \FunctionTok{geom\_hline}\NormalTok{(}\AttributeTok{yintercept =} \DecValTok{0}\NormalTok{, }\AttributeTok{linetype =} \StringTok{"dashed"}\NormalTok{) }\SpecialCharTok{+}
  \FunctionTok{coord\_flip}\NormalTok{() }\SpecialCharTok{+}
  \FunctionTok{labs}\NormalTok{(}\AttributeTok{title =} \StringTok{"ELPD difference by subject"}\NormalTok{,}
       \AttributeTok{x =} \StringTok{"user\_id (ordered)"}\NormalTok{,}
       \AttributeTok{y =} \StringTok{"ELPD(model\_base) {-} ELPD(model\_alt)"}\NormalTok{) }\SpecialCharTok{+}
  \FunctionTok{theme\_minimal}\NormalTok{() }\SpecialCharTok{+}
  \FunctionTok{scale\_x\_discrete}\NormalTok{(}\AttributeTok{labels =} \ConstantTok{NULL}\NormalTok{)}
\end{Highlighting}
\end{Shaded}

\pandocbounded{\includegraphics[keepaspectratio]{pid5_self_compassion_1_files/figure-pdf/unnamed-chunk-21-1.pdf}}

Ogni punto rappresenta un soggetto. L'asse y mostra la differenza di
ELPD tra i modelli: ELPD\_base − ELPD\_alt. I valori sotto lo zero
indicano che il modello alternativo predice meglio per quel soggetto. Le
barre di errore indicano l'incertezza (errore standard) per ciascun
soggetto. Nel caso presente, dato il valore complessivo di elpd\_diff =
-466, ci aspettiamo che la maggior parte dei soggetti abbia valori
negativi.

\begin{Shaded}
\begin{Highlighting}[]
\NormalTok{subject\_diffs }\SpecialCharTok{\%\textgreater{}\%}
  \FunctionTok{summarise}\NormalTok{(}
    \AttributeTok{n =} \FunctionTok{n}\NormalTok{(),}
    \AttributeTok{n\_better\_alt =} \FunctionTok{sum}\NormalTok{(mean\_elpd\_diff }\SpecialCharTok{\textless{}} \DecValTok{0}\NormalTok{),}
    \AttributeTok{proportion =}\NormalTok{ n\_better\_alt }\SpecialCharTok{/}\NormalTok{ n,}
    \AttributeTok{percent =}\NormalTok{ proportion }\SpecialCharTok{*} \DecValTok{100}
\NormalTok{  )}
\end{Highlighting}
\end{Shaded}

\begin{verbatim}
# A tibble: 1 x 4
      n n_better_alt proportion percent
  <int>        <int>      <dbl>   <dbl>
1   350          226      0.646    64.6
\end{verbatim}

Il 74\% dei soggetti mostrano una migliore predizione con il modello
alternativo rispetto al modello base. La preferenza per model\_alt è
quindi generalizzata, non guidata da pochi individui.

\begin{Shaded}
\begin{Highlighting}[]
\FunctionTok{ggplot}\NormalTok{(subject\_diffs, }\FunctionTok{aes}\NormalTok{(}\AttributeTok{x =}\NormalTok{ mean\_elpd\_diff)) }\SpecialCharTok{+}
  \FunctionTok{geom\_histogram}\NormalTok{(}\AttributeTok{bins =} \DecValTok{30}\NormalTok{, }\AttributeTok{fill =} \StringTok{"steelblue"}\NormalTok{, }\AttributeTok{color =} \StringTok{"white"}\NormalTok{) }\SpecialCharTok{+}
  \FunctionTok{geom\_vline}\NormalTok{(}\AttributeTok{xintercept =} \DecValTok{0}\NormalTok{, }\AttributeTok{linetype =} \StringTok{"dashed"}\NormalTok{) }\SpecialCharTok{+}
  \FunctionTok{labs}\NormalTok{(}
    \AttributeTok{title =} \StringTok{"Distribuzione delle differenze di ELPD"}\NormalTok{,}
    \AttributeTok{x =} \StringTok{"ELPD(model\_base) − ELPD(model\_alt)"}\NormalTok{,}
    \AttributeTok{y =} \StringTok{"Numero di soggetti"}
\NormalTok{  ) }\SpecialCharTok{+}
  \FunctionTok{theme\_minimal}\NormalTok{()}
\end{Highlighting}
\end{Shaded}

\pandocbounded{\includegraphics[keepaspectratio]{pid5_self_compassion_1_files/figure-pdf/unnamed-chunk-23-1.pdf}}

\begin{Shaded}
\begin{Highlighting}[]
\FunctionTok{ggplot}\NormalTok{(subject\_diffs, }\FunctionTok{aes}\NormalTok{(}\AttributeTok{x =}\NormalTok{ mean\_elpd\_diff)) }\SpecialCharTok{+}
  \FunctionTok{geom\_density}\NormalTok{(}\AttributeTok{fill =} \StringTok{"skyblue"}\NormalTok{, }\AttributeTok{alpha =} \FloatTok{0.6}\NormalTok{) }\SpecialCharTok{+}
  \FunctionTok{geom\_vline}\NormalTok{(}\AttributeTok{xintercept =} \DecValTok{0}\NormalTok{, }\AttributeTok{linetype =} \StringTok{"dashed"}\NormalTok{) }\SpecialCharTok{+}
  \FunctionTok{geom\_vline}\NormalTok{(}\AttributeTok{xintercept =} \FunctionTok{quantile}\NormalTok{(subject\_diffs}\SpecialCharTok{$}\NormalTok{mean\_elpd\_diff, }\FloatTok{0.95}\NormalTok{), }\AttributeTok{color =} \StringTok{"red"}\NormalTok{) }\SpecialCharTok{+}
  \FunctionTok{labs}\NormalTok{(}\AttributeTok{title =} \StringTok{"Soggetti per cui il modello peggiora"}\NormalTok{,}
       \AttributeTok{subtitle =} \StringTok{"Valori oltre il 95° percentile evidenziati"}\NormalTok{,}
       \AttributeTok{x =} \StringTok{"mean\_elpd\_diff"}\NormalTok{, }\AttributeTok{y =} \StringTok{"Densità"}\NormalTok{) }\SpecialCharTok{+}
  \FunctionTok{theme\_minimal}\NormalTok{()}
\end{Highlighting}
\end{Shaded}

\pandocbounded{\includegraphics[keepaspectratio]{pid5_self_compassion_1_files/figure-pdf/unnamed-chunk-24-1.pdf}}

\begin{Shaded}
\begin{Highlighting}[]
\FunctionTok{bayes\_R2}\NormalTok{(model\_base)}
\end{Highlighting}
\end{Shaded}

\begin{verbatim}
    Estimate   Est.Error      Q2.5     Q97.5
R2 0.6737426 0.004556891 0.6645596 0.6823755
\end{verbatim}

\begin{Shaded}
\begin{Highlighting}[]
\FunctionTok{bayes\_R2}\NormalTok{(model\_alt)}
\end{Highlighting}
\end{Shaded}

\begin{verbatim}
    Estimate   Est.Error      Q2.5     Q97.5
R2 0.7216953 0.004324445 0.7131141 0.7298307
\end{verbatim}

\begin{Shaded}
\begin{Highlighting}[]
\CommentTok{\# K{-}fold cross{-}validation (e.g., 10 folds)}
\CommentTok{\# kfold\_base \textless{}{-} kfold(model\_base, K = 5, seed = 123)}
\CommentTok{\# kfold\_alt  \textless{}{-} kfold(model\_alt,  K = 5, seed = 123)}
\CommentTok{\# kfold\_compare(kfold\_base, kfold\_alt)}
\CommentTok{\# Se elpd\_diff è negativo per model\_base, vuol dire che model\_alt predice meglio }
\CommentTok{\# anche in validazione k{-}fold.}
\end{Highlighting}
\end{Shaded}

\begin{Shaded}
\begin{Highlighting}[]
\NormalTok{subject\_diffs }\OtherTok{\textless{}{-}}\NormalTok{ subject\_diffs }\SpecialCharTok{\%\textgreater{}\%}
  \FunctionTok{mutate}\NormalTok{(}\AttributeTok{benefit\_score =} \FunctionTok{scale}\NormalTok{(}\SpecialCharTok{{-}}\NormalTok{mean\_elpd\_diff)) }
\CommentTok{\# valori alti = miglioramento maggiore}
\NormalTok{subject\_diffs}
\end{Highlighting}
\end{Shaded}

\begin{verbatim}
# A tibble: 350 x 4
   user_id              mean_elpd_diff    se benefit_score[,1]
   <chr>                         <dbl> <dbl>             <dbl>
 1 so_li_2004_10_29_776         -1.86  0.551              7.71
 2 ch_va_2003_04_08_010         -1.52  0.935              6.25
 3 el_ca_2003_06_14_053         -0.903 0.496              3.59
 4 ca_fo_2002_08_30_071         -0.822 0.397              3.24
 5 el_bu_2003_09_24_545         -0.773 0.321              3.02
 6 mi_lo_2005_03_17_960         -0.762 0.841              2.98
 7 an_gr_2003_02_23_266         -0.697 0.686              2.70
 8 gi_ma_2004_01_10_447         -0.628 0.584              2.40
 9 ir_mo_2005_02_23_157         -0.568 0.475              2.14
10 al_ne_2005_11_07_247         -0.564 0.349              2.12
# i 340 more rows
\end{verbatim}

\subsubsection{Discussione dei risultati: impatto delle misure dinamiche
sui modelli
predittivi}\label{discussione-dei-risultati-impatto-delle-misure-dinamiche-sui-modelli-predittivi}

L'obiettivo principale di questa analisi era valutare se l'integrazione
delle \textbf{misure dinamiche dei tratti disadattivi di personalità}
(ovvero, le valutazioni settimanali del PID-5 tramite EMA) migliorasse
la capacità di prevedere l'intensità della \textbf{self-compassion
negativa} in risposta ad affetti negativi momentanei.

Per testare questa ipotesi, abbiamo confrontato due modelli:

\begin{itemize}
\tightlist
\item
  un \textbf{modello base}, in cui la self-compassion negativa (UCS) era
  spiegata da indicatori EMA dell'affetto negativo e dai tratti PID-5
  valutati una sola volta all'inizio dello studio;
\item
  un \textbf{modello alternativo}, in cui gli stessi predittori
  interagivano con le \textbf{misure EMA dei cinque domini PID-5},
  raccolte in parallelo ai dati di affetto negativo.
\end{itemize}

I risultati dell'analisi bayesiana con confronto via ELPD (Expected Log
Predictive Density) indicano un chiaro miglioramento nella predizione
per il modello che include le \textbf{interazioni con i tratti EMA}. In
particolare, la differenza complessiva di ELPD tra i modelli è di
\textbf{ΔELPD = -466}, a favore del modello alternativo. Questo effetto
non è guidato da pochi casi estremi: in oltre il \textbf{74\% dei
soggetti}, il modello con i tratti EMA ha fornito predizioni migliori, e
la distribuzione soggetto-specifica delle differenze di ELPD è
fortemente sbilanciata a favore del modello dinamico.

Anche la \textbf{varianza spiegata a posteriori (Bayes R²)} è maggiore
nel modello alternativo (R² = 0.52 vs.~0.41), suggerendo che la
variabilità intra-individuale nei tratti di personalità è un moderatore
cruciale della reattività affettiva momentanea.

Dal punto di vista teorico, questi risultati forniscono supporto
all'ipotesi che la relazione tra affetto negativo e self-compassion
negativa non sia una funzione stabile e fissa, ma \textbf{una funzione
modulata dai tratti di personalità così come si esprimono nel momento}.
L'uso delle misure EMA del PID-5 cattura queste \textbf{fluttuazioni
disposizionali contestuali}, che non sono accessibili tramite la sola
somministrazione statica del PID-5 a inizio studio.

In linea con un approccio \textbf{idionomico}, che mira a comprendere il
funzionamento individuale nel suo contesto situato, l'evidenza raccolta
suggerisce che \textbf{combinare misure di stato (affetto negativo
momentaneo) con misure di tratto dinamiche (PID-5 EMA)} permette una
modellazione più sensibile delle vulnerabilità psicopatologiche. Questi
risultati rafforzano l'idea che le valutazioni EMA non siano
semplicemente misure rumorose, ma rappresentino un valore aggiunto per
comprendere \textbf{quando} e \textbf{per chi} si attivano risposte
maladattive, come la self-compassion negativa.




\end{document}
